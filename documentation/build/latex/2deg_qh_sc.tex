%% Generated by Sphinx.
\def\sphinxdocclass{report}
\documentclass[letterpaper,10pt,english]{sphinxmanual}
\ifdefined\pdfpxdimen
   \let\sphinxpxdimen\pdfpxdimen\else\newdimen\sphinxpxdimen
\fi \sphinxpxdimen=.75bp\relax
\ifdefined\pdfimageresolution
    \pdfimageresolution= \numexpr \dimexpr1in\relax/\sphinxpxdimen\relax
\fi
%% let collapsible pdf bookmarks panel have high depth per default
\PassOptionsToPackage{bookmarksdepth=5}{hyperref}

\PassOptionsToPackage{warn}{textcomp}
\usepackage[utf8]{inputenc}
\ifdefined\DeclareUnicodeCharacter
% support both utf8 and utf8x syntaxes
  \ifdefined\DeclareUnicodeCharacterAsOptional
    \def\sphinxDUC#1{\DeclareUnicodeCharacter{"#1}}
  \else
    \let\sphinxDUC\DeclareUnicodeCharacter
  \fi
  \sphinxDUC{00A0}{\nobreakspace}
  \sphinxDUC{2500}{\sphinxunichar{2500}}
  \sphinxDUC{2502}{\sphinxunichar{2502}}
  \sphinxDUC{2514}{\sphinxunichar{2514}}
  \sphinxDUC{251C}{\sphinxunichar{251C}}
  \sphinxDUC{2572}{\textbackslash}
\fi
\usepackage{cmap}
\usepackage[T1]{fontenc}
\usepackage{amsmath,amssymb,amstext}
\usepackage{babel}



\usepackage{tgtermes}
\usepackage{tgheros}
\renewcommand{\ttdefault}{txtt}



\usepackage[Bjarne]{fncychap}
\usepackage{sphinx}

\fvset{fontsize=auto}
\usepackage{geometry}


% Include hyperref last.
\usepackage{hyperref}
% Fix anchor placement for figures with captions.
\usepackage{hypcap}% it must be loaded after hyperref.
% Set up styles of URL: it should be placed after hyperref.
\urlstyle{same}

\addto\captionsenglish{\renewcommand{\contentsname}{Contents}}

\usepackage{sphinxmessages}
\setcounter{tocdepth}{1}



\title{2deg\_QH\_SC}
\date{Oct 10, 2023}
\release{3.0}
\author{A.\@{} David, J.\@{} S.\@{} Meyer, M.\@{} Houzet}
\newcommand{\sphinxlogo}{\vbox{}}
\renewcommand{\releasename}{Release}
\makeindex
\begin{document}

\pagestyle{empty}
\sphinxmaketitle
\pagestyle{plain}
\sphinxtableofcontents
\pagestyle{normal}
\phantomsection\label{\detokenize{index::doc}}


\sphinxAtStartPar
Documentation of the code used in the paper “Geometrical effects on the downstream conductance in quantum\sphinxhyphen{}Hall\textendash{}superconductor hybrid systems” available on \sphinxhref{https://arxiv.org/abs/2210.16867}{arXiv} and published in \sphinxhref{https://journals.aps.org/prb/abstract/10.1103/PhysRevB.107.125416}{Phys. Rev. B}. The code is available in a \sphinxhref{https://github.com/akdavid/2deg\_QH\_SC/}{Github Repository}.

\sphinxAtStartPar
Check the {\hyperref[\detokenize{usage::doc}]{\sphinxcrossref{\DUrole{doc}{Installation and Usage}}}} section to install and use the code. The {\hyperref[\detokenize{main_scripts::doc}]{\sphinxcrossref{\DUrole{doc}{Main Scripts}}}} section includes the script \sphinxstyleemphasis{calculations.py} which can be used to run the different calculations and the script \sphinxstyleemphasis{manuscript\_figures.py} that generates the manuscript figures. The {\hyperref[\detokenize{modules::doc}]{\sphinxcrossref{\DUrole{doc}{Modules}}}} section describes the different modules used to run the calculations.

\sphinxstepscope


\chapter{Installation and Usage}
\label{\detokenize{usage:installation-and-usage}}\label{\detokenize{usage::doc}}
\sphinxAtStartPar
Here we explain how to run the scripts.
After following the steps below try to run some calculations
with the script \sphinxstyleemphasis{calculations.py} or reproduce the manuscript
figures with the script \sphinxstyleemphasis{manuscript\_figures.py}.


\section{Miniconda, GitHub repository and conda environment}
\label{\detokenize{usage:miniconda-github-repository-and-conda-environment}}
\sphinxAtStartPar
Install \sphinxhref{http://conda.pydata.org/miniconda.html/}{miniconda}
, clone the \sphinxhref{https://github.com/akdavid/2deg\_QH-SC/}{repository}
and \sphinxcode{\sphinxupquote{cd}} into the root directory \sphinxstyleemphasis{2deg\_QH\sphinxhyphen{}SC\sphinxhyphen{}main} after unzipping.
Then create the Conda environment that contains all dependencies with

\begin{sphinxVerbatim}[commandchars=\\\{\}]
\PYG{n}{conda} \PYG{n}{env} \PYG{n}{create} \PYG{o}{\PYGZhy{}}\PYG{n}{f} \PYG{n}{environment}\PYG{o}{.}\PYG{n}{yml}
\end{sphinxVerbatim}

\sphinxAtStartPar
You can now use this environment to run the scripts.
Below is detailed how to run the scripts using the command line,
Pycharm or Visual Studio Code.


\section{Running the scripts by using the command line}
\label{\detokenize{usage:running-the-scripts-by-using-the-command-line}}
\sphinxAtStartPar
To run a script from the terminal use the following command line:

\begin{sphinxVerbatim}[commandchars=\\\{\}]
\PYG{o}{\PYGZlt{}}\PYG{n}{path\PYGZus{}to\PYGZus{}python\PYGZus{}exe}\PYG{o}{\PYGZgt{}} \PYG{o}{\PYGZlt{}}\PYG{n}{path\PYGZus{}to\PYGZus{}python\PYGZus{}script}\PYG{o}{\PYGZgt{}}
\end{sphinxVerbatim}

\sphinxAtStartPar
The python executable  should be located at :
\begin{quote}

\sphinxAtStartPar
\textasciitilde{}/opt/miniconda3/envs/2deg\_QH\sphinxhyphen{}SC/bin/python (macOS)

\sphinxAtStartPar
\textasciitilde{}/miniconda3/envs/2deg\_QH\sphinxhyphen{}SC/bin/python (Linux)

\sphinxAtStartPar
\textasciitilde{}/miniconda3/envs/2deg\_QH\sphinxhyphen{}SC/python (Windows)
\end{quote}

\sphinxAtStartPar
For example, if you are on macOS and you want to run the script
\sphinxstyleemphasis{calculations.py} (while being in the root directory of the repo), use

\begin{sphinxVerbatim}[commandchars=\\\{\}]
\PYG{o}{\PYGZti{}}\PYG{o}{/}\PYG{n}{opt}\PYG{o}{/}\PYG{n}{miniconda3}\PYG{o}{/}\PYG{n}{envs}\PYG{o}{/}\PYG{l+m+mi}{2}\PYG{n}{deg\PYGZus{}QH}\PYG{o}{\PYGZhy{}}\PYG{n}{SC}\PYG{o}{/}\PYG{n+nb}{bin}\PYG{o}{/}\PYG{n}{python} \PYG{n}{calculations}\PYG{o}{.}\PYG{n}{py}
\end{sphinxVerbatim}

\begin{sphinxadmonition}{note}{Note:}
\sphinxAtStartPar
You can use a global shell variable to create a shortcut to the Python executable path.
For that, open a terminal and modify the bash configuration file located in your
HOME directory (the directory in which you are when you open the terminal)
(.bashrc, .bash\_profile, or .profile). For example on macOS

\begin{sphinxVerbatim}[commandchars=\\\{\}]
\PYG{n}{nano} \PYG{o}{.}\PYG{n}{bash\PYGZus{}profile}
\end{sphinxVerbatim}

\sphinxAtStartPar
Add the following in the file

\begin{sphinxVerbatim}[commandchars=\\\{\}]
\PYG{n}{export} \PYG{n}{mypython}\PYG{o}{=}\PYG{o}{\PYGZti{}}\PYG{o}{/}\PYG{n}{opt}\PYG{o}{/}\PYG{n}{miniconda3}\PYG{o}{/}\PYG{n}{envs}\PYG{o}{/}\PYG{l+m+mi}{2}\PYG{n}{deg\PYGZus{}QH}\PYG{o}{\PYGZhy{}}\PYG{n}{SC}\PYG{o}{/}\PYG{n+nb}{bin}\PYG{o}{/}\PYG{n}{python}
\end{sphinxVerbatim}

\sphinxAtStartPar
and save it with \sphinxcode{\sphinxupquote{Ctrl+X}} and \sphinxcode{\sphinxupquote{Y}} and \sphinxcode{\sphinxupquote{Enter}}.
Then close the terminal and open a new one to make the modification effective.
You can now use the variable \sphinxcode{\sphinxupquote{\$mypython}}
for the path such that the above example reads

\begin{sphinxVerbatim}[commandchars=\\\{\}]
\PYGZdl{}mypython calculations.py
\end{sphinxVerbatim}
\end{sphinxadmonition}


\section{Running the scripts by using PyCharm}
\label{\detokenize{usage:running-the-scripts-by-using-pycharm}}
\sphinxAtStartPar
With \sphinxhref{https://www.jetbrains.com/pycharm/download/}{PyCharm} you can follow these steps :
\begin{enumerate}
\sphinxsetlistlabels{\arabic}{enumi}{enumii}{}{.}%
\item {} 
\sphinxAtStartPar
Launch PyCharm and choose \sphinxcode{\sphinxupquote{Create New Project}}

\item {} 
\sphinxAtStartPar
\sphinxstylestrong{Locate the project} at the root directory \sphinxstyleemphasis{2deg\_QH\sphinxhyphen{}SC\sphinxhyphen{}main}.

\item {} 
\sphinxAtStartPar
Mark \sphinxcode{\sphinxupquote{Existing interpreter}} (or \sphinxcode{\sphinxupquote{Previously configured interpreter}})
and click on the selection icon \sphinxcode{\sphinxupquote{...}}

\item {} 
\sphinxAtStartPar
Select \sphinxcode{\sphinxupquote{Conda Environment}} and choose the location of the python executable.
It should be located at :
\begin{quote}

\sphinxAtStartPar
\textasciitilde{}/opt/miniconda3/envs/2deg\_QH\sphinxhyphen{}SC/bin/python (macOS)

\sphinxAtStartPar
\textasciitilde{}/miniconda3/envs/2deg\_QH\sphinxhyphen{}SC/bin/python (Linux)

\sphinxAtStartPar
\textasciitilde{}/miniconda3/envs/2deg\_QH\sphinxhyphen{}SC/python (Windows)
\end{quote}

\item {} 
\sphinxAtStartPar
Click on \sphinxcode{\sphinxupquote{Ok}} then on \sphinxcode{\sphinxupquote{Create}} and select \sphinxcode{\sphinxupquote{Create from existing sources}}

\item {} 
\sphinxAtStartPar
You are ready to run the scripts!

\item {} 
\sphinxAtStartPar
(Optional) You can see progress bars during the calculations by activating the
\sphinxcode{\sphinxupquote{Emulate terminal in output console}} option. For that, got to \sphinxcode{\sphinxupquote{Run\sphinxhyphen{}\textgreater{}Edit Configurations}}
and check the option.

\end{enumerate}


\section{Running the scripts by using Visual Studio Code}
\label{\detokenize{usage:running-the-scripts-by-using-visual-studio-code}}
\sphinxAtStartPar
With \sphinxhref{https://code.visualstudio.com/download/}{Visual Studio Code} you can follow these steps :
\begin{enumerate}
\sphinxsetlistlabels{\arabic}{enumi}{enumii}{}{.}%
\item {} 
\sphinxAtStartPar
Launch Visual Studio Code and install the \sphinxstyleemphasis{Python} extension if it’s not done yet.

\item {} 
\sphinxAtStartPar
From the main page choose \sphinxcode{\sphinxupquote{Open...}} , select the root directory \sphinxstyleemphasis{2deg\_QH\sphinxhyphen{}SC\sphinxhyphen{}main}, and click on
\sphinxcode{\sphinxupquote{Yes, I trust the authors}}

\item {} 
\sphinxAtStartPar
Open the \sphinxstyleemphasis{Command Palette} with \sphinxcode{\sphinxupquote{Ctrl+Shift+P}}, search
\sphinxcode{\sphinxupquote{Python: Select Interpreter}} and choose the one associated to the ‘2deg\_QH\sphinxhyphen{}SC’
environment.

\item {} 
\sphinxAtStartPar
You are ready to run the scripts!

\end{enumerate}


\section{Updating the documentation}
\label{\detokenize{usage:updating-the-documentation}}
\sphinxAtStartPar
The \sphinxstyleemphasis{documentation} directory contains a \sphinxcode{\sphinxupquote{pdf}} and a local \sphinxcode{\sphinxupquote{html}} version of the documentation.
They can respectively be found at \sphinxstyleemphasis{documentation/build/latex/2deg\_qh\sphinxhyphen{}sc.pdf}
and \sphinxstyleemphasis{documentation/build/html/index.html}. You can update them by using

\begin{sphinxVerbatim}[commandchars=\\\{\}]
\PYG{n}{conda} \PYG{n}{activate} \PYG{l+m+mi}{2}\PYG{n}{deg\PYGZus{}QH}\PYG{o}{\PYGZhy{}}\PYG{n}{SC}
\PYG{n}{cd} \PYG{n}{documentation}
\PYG{n}{make} \PYG{n}{html}
\PYG{n}{make} \PYG{n}{latexpdf}
\end{sphinxVerbatim}

\begin{sphinxadmonition}{note}{Note:}
\sphinxAtStartPar
When you compile the documentation it runs the python scripts so make sure
the calculations are commented before using \sphinxcode{\sphinxupquote{make html}} or \sphinxcode{\sphinxupquote{make latexpdf}}.
\end{sphinxadmonition}

\sphinxstepscope


\chapter{Main Scripts}
\label{\detokenize{main_scripts:main-scripts}}\label{\detokenize{main_scripts::doc}}
\sphinxAtStartPar
The scripts present in the \sphinxtitleref{main\_code} directory are described in this section.


\section{calculations.py}
\label{\detokenize{main_scripts:module-calculations}}\label{\detokenize{main_scripts:calculations-py}}\index{module@\spxentry{module}!calculations@\spxentry{calculations}}\index{calculations@\spxentry{calculations}!module@\spxentry{module}}
\sphinxAtStartPar
Calculations.

\sphinxAtStartPar
This script contains different sections, each of which 
generates the data and the plot of a calculation. Comment the quotation
marks by using hashtags to run the calculations.

\sphinxAtStartPar
Contents:
\begin{itemize}
\item {} 
\sphinxAtStartPar
Visualization
\begin{itemize}
\item {} 
\sphinxAtStartPar
Kwant system

\item {} 
\sphinxAtStartPar
Density u\textasciicircum{}2 \sphinxhyphen{} v\textasciicircum{}2

\end{itemize}

\item {} 
\sphinxAtStartPar
Energy Spectrum
\begin{itemize}
\item {} 
\sphinxAtStartPar
Tight\sphinxhyphen{}binding spectrum

\item {} 
\sphinxAtStartPar
Microscopic spectrum

\item {} 
\sphinxAtStartPar
Spectrum comparison

\end{itemize}

\item {} 
\sphinxAtStartPar
Momentum at the Fermi level k0
\begin{itemize}
\item {} 
\sphinxAtStartPar
k0 \sphinxstyleemphasis{v.s.} nu

\item {} 
\sphinxAtStartPar
k0 \sphinxstyleemphasis{v.s.} Z at various fillings

\end{itemize}

\item {} 
\sphinxAtStartPar
Andreev Transmission and Hole Probability
\begin{itemize}
\item {} 
\sphinxAtStartPar
tau \sphinxstyleemphasis{v.s.} theta\_qh at various fillings

\item {} 
\sphinxAtStartPar
tau \sphinxstyleemphasis{v.s.} theta\_sc at various fillings

\item {} 
\sphinxAtStartPar
tau \sphinxstyleemphasis{v.s.} mu\_qh/Delta at various fillings

\item {} 
\sphinxAtStartPar
fh\_p \sphinxstyleemphasis{v.s.} Z at various fillings

\end{itemize}

\item {} 
\sphinxAtStartPar
Downstream conductance
\begin{itemize}
\item {} 
\sphinxAtStartPar
Conductance comparison

\end{itemize}

\item {} 
\sphinxAtStartPar
Track states
\begin{itemize}
\item {} 
\sphinxAtStartPar
nu\_crit \sphinxstyleemphasis{v.s.} mu\_qh/delta

\item {} 
\sphinxAtStartPar
Asymptotic nu\_crit \sphinxstyleemphasis{v.s.} mu\_sc/mu\_qh

\item {} 
\sphinxAtStartPar
Asymptotic nu\_crit \sphinxstyleemphasis{v.s.} Z

\end{itemize}

\item {} 
\sphinxAtStartPar
Finite\sphinxhyphen{}temperature
\begin{itemize}
\item {} 
\sphinxAtStartPar
Momentum difference \sphinxstyleemphasis{v.s.} energy

\item {} 
\sphinxAtStartPar
tau \sphinxstyleemphasis{v.s.} energy

\item {} 
\sphinxAtStartPar
Finite\sphinxhyphen{}temperature tight\sphinxhyphen{}binding conductance \sphinxstyleemphasis{v.s.} L

\item {} 
\sphinxAtStartPar
Finite\sphinxhyphen{}temperature tight\sphinxhyphen{}binding conductance \sphinxstyleemphasis{v.s.} L at various temperatures

\end{itemize}

\end{itemize}

\sphinxAtStartPar
The resulting data and plots are saved in the ‘files‘ directory.


\section{manuscript\_figures.py}
\label{\detokenize{main_scripts:module-manuscript_figures}}\label{\detokenize{main_scripts:manuscript-figures-py}}\index{module@\spxentry{module}!manuscript\_figures@\spxentry{manuscript\_figures}}\index{manuscript\_figures@\spxentry{manuscript\_figures}!module@\spxentry{module}}
\sphinxAtStartPar
This script generates the figures as used in the manuscript.

\sphinxAtStartPar
To re\sphinxhyphen{}compute the data use the option from\_data=False in the plot functions.
Due to the option fig\_name the plots are saved in the ‘figures‘ directory.

\sphinxstepscope


\chapter{Modules}
\label{\detokenize{modules:modules}}\label{\detokenize{modules::doc}}
\sphinxAtStartPar
The different modules are described in this section.


\section{functions.py}
\label{\detokenize{modules:module-modules.functions}}\label{\detokenize{modules:functions-py}}\index{module@\spxentry{module}!modules.functions@\spxentry{modules.functions}}\index{modules.functions@\spxentry{modules.functions}!module@\spxentry{module}}
\sphinxAtStartPar
Definition of the functions required for the calculations.
\index{U() (in module modules.functions)@\spxentry{U()}\spxextra{in module modules.functions}}

\begin{fulllineitems}
\phantomsection\label{\detokenize{modules:modules.functions.U}}
\pysigstartsignatures
\pysiglinewithargsret{\sphinxcode{\sphinxupquote{modules.functions.}}\sphinxbfcode{\sphinxupquote{U}}}{\emph{\DUrole{n}{n}}, \emph{\DUrole{n}{z}}}{}
\pysigstopsignatures
\sphinxAtStartPar
Definition of the parabolic cylinder function U(n, z) as defined in Abramowitz \& Stegun book.
\begin{quote}\begin{description}
\item[{Parameters}] \leavevmode\begin{itemize}
\item {} 
\sphinxAtStartPar
\sphinxstyleliteralstrong{\sphinxupquote{n}} (\sphinxstyleliteralemphasis{\sphinxupquote{float}}) \textendash{} First argument of the parabolic cylinder function.

\item {} 
\sphinxAtStartPar
\sphinxstyleliteralstrong{\sphinxupquote{z}} (\sphinxstyleliteralemphasis{\sphinxupquote{float}}) \textendash{} Second argument of the parabolic cylinder function.

\end{itemize}

\item[{Returns}] \leavevmode
\sphinxAtStartPar
The value of the parabolic cylinder function U(n, z).

\item[{Return type}] \leavevmode
\sphinxAtStartPar
float

\end{description}\end{quote}

\end{fulllineitems}

\index{U\_m() (in module modules.functions)@\spxentry{U\_m()}\spxextra{in module modules.functions}}

\begin{fulllineitems}
\phantomsection\label{\detokenize{modules:modules.functions.U_m}}
\pysigstartsignatures
\pysiglinewithargsret{\sphinxcode{\sphinxupquote{modules.functions.}}\sphinxbfcode{\sphinxupquote{U\_m}}}{\emph{\DUrole{n}{x}}, \emph{\DUrole{n}{E}}, \emph{\DUrole{n}{k}}, \emph{\DUrole{n}{m\_qh}}, \emph{\DUrole{n}{mu\_qh}}, \emph{\DUrole{n}{omega}}}{}
\pysigstopsignatures
\sphinxAtStartPar
Hole\sphinxhyphen{}like solution in QH region.
\begin{quote}\begin{description}
\item[{Parameters}] \leavevmode\begin{itemize}
\item {} 
\sphinxAtStartPar
\sphinxstyleliteralstrong{\sphinxupquote{x}} (\sphinxstyleliteralemphasis{\sphinxupquote{float}}) \textendash{} x\sphinxhyphen{}coordinate.

\item {} 
\sphinxAtStartPar
\sphinxstyleliteralstrong{\sphinxupquote{E}} (\sphinxstyleliteralemphasis{\sphinxupquote{float}}) \textendash{} Energy measured from the Fermi level.

\item {} 
\sphinxAtStartPar
\sphinxstyleliteralstrong{\sphinxupquote{k}} (\sphinxstyleliteralemphasis{\sphinxupquote{float}}) \textendash{} Momentum along the QH\sphinxhyphen{}SC interface.

\item {} 
\sphinxAtStartPar
\sphinxstyleliteralstrong{\sphinxupquote{m\_qh}} (\sphinxstyleliteralemphasis{\sphinxupquote{float}}) \textendash{} Effective mass in the QH region.

\item {} 
\sphinxAtStartPar
\sphinxstyleliteralstrong{\sphinxupquote{mu\_qh}} (\sphinxstyleliteralemphasis{\sphinxupquote{float}}) \textendash{} Chemical potential in the QH region.

\item {} 
\sphinxAtStartPar
\sphinxstyleliteralstrong{\sphinxupquote{omega}} (\sphinxstyleliteralemphasis{\sphinxupquote{float}}) \textendash{} Cyclotron frequency.

\end{itemize}

\item[{Returns}] \leavevmode
\sphinxAtStartPar
The value of the parabolic cylinder function associated to holes.

\item[{Return type}] \leavevmode
\sphinxAtStartPar
float

\end{description}\end{quote}

\end{fulllineitems}

\index{U\_p() (in module modules.functions)@\spxentry{U\_p()}\spxextra{in module modules.functions}}

\begin{fulllineitems}
\phantomsection\label{\detokenize{modules:modules.functions.U_p}}
\pysigstartsignatures
\pysiglinewithargsret{\sphinxcode{\sphinxupquote{modules.functions.}}\sphinxbfcode{\sphinxupquote{U\_p}}}{\emph{\DUrole{n}{x}}, \emph{\DUrole{n}{E}}, \emph{\DUrole{n}{k}}, \emph{\DUrole{n}{m\_qh}}, \emph{\DUrole{n}{mu\_qh}}, \emph{\DUrole{n}{omega}}}{}
\pysigstopsignatures
\sphinxAtStartPar
Electron\sphinxhyphen{}like solution in QH region.
\begin{quote}\begin{description}
\item[{Parameters}] \leavevmode\begin{itemize}
\item {} 
\sphinxAtStartPar
\sphinxstyleliteralstrong{\sphinxupquote{x}} (\sphinxstyleliteralemphasis{\sphinxupquote{float}}) \textendash{} x\sphinxhyphen{}coordinate.

\item {} 
\sphinxAtStartPar
\sphinxstyleliteralstrong{\sphinxupquote{E}} (\sphinxstyleliteralemphasis{\sphinxupquote{float}}) \textendash{} Energy measured from the Fermi level.

\item {} 
\sphinxAtStartPar
\sphinxstyleliteralstrong{\sphinxupquote{k}} (\sphinxstyleliteralemphasis{\sphinxupquote{float}}) \textendash{} Momentum along the QH\sphinxhyphen{}SC interface.

\item {} 
\sphinxAtStartPar
\sphinxstyleliteralstrong{\sphinxupquote{m\_qh}} (\sphinxstyleliteralemphasis{\sphinxupquote{float}}) \textendash{} Effective mass in the QH region.

\item {} 
\sphinxAtStartPar
\sphinxstyleliteralstrong{\sphinxupquote{mu\_qh}} (\sphinxstyleliteralemphasis{\sphinxupquote{float}}) \textendash{} Chemical potential in the QH region.

\item {} 
\sphinxAtStartPar
\sphinxstyleliteralstrong{\sphinxupquote{omega}} (\sphinxstyleliteralemphasis{\sphinxupquote{float}}) \textendash{} Cyclotron frequency.

\end{itemize}

\item[{Returns}] \leavevmode
\sphinxAtStartPar
The value of the parabolic cylinder function associated to electrons.

\item[{Return type}] \leavevmode
\sphinxAtStartPar
float

\end{description}\end{quote}

\end{fulllineitems}

\index{chi\_m() (in module modules.functions)@\spxentry{chi\_m()}\spxextra{in module modules.functions}}

\begin{fulllineitems}
\phantomsection\label{\detokenize{modules:modules.functions.chi_m}}
\pysigstartsignatures
\pysiglinewithargsret{\sphinxcode{\sphinxupquote{modules.functions.}}\sphinxbfcode{\sphinxupquote{chi\_m}}}{\emph{\DUrole{n}{x}}, \emph{\DUrole{n}{E}}, \emph{\DUrole{n}{k}}, \emph{\DUrole{n}{m\_qh}}, \emph{\DUrole{n}{mu\_qh}}, \emph{\DUrole{n}{nu}}}{}
\pysigstopsignatures
\sphinxAtStartPar
Hole\sphinxhyphen{}like wave function in QH region.
\begin{quote}\begin{description}
\item[{Parameters}] \leavevmode\begin{itemize}
\item {} 
\sphinxAtStartPar
\sphinxstyleliteralstrong{\sphinxupquote{x}} (\sphinxstyleliteralemphasis{\sphinxupquote{float}}) \textendash{} x\sphinxhyphen{}coordinate.

\item {} 
\sphinxAtStartPar
\sphinxstyleliteralstrong{\sphinxupquote{E}} (\sphinxstyleliteralemphasis{\sphinxupquote{float}}) \textendash{} Energy measured from the Fermi level.

\item {} 
\sphinxAtStartPar
\sphinxstyleliteralstrong{\sphinxupquote{k}} (\sphinxstyleliteralemphasis{\sphinxupquote{float}}) \textendash{} Momentum along the QH\sphinxhyphen{}SC interface.

\item {} 
\sphinxAtStartPar
\sphinxstyleliteralstrong{\sphinxupquote{m\_qh}} (\sphinxstyleliteralemphasis{\sphinxupquote{float}}) \textendash{} Effective mass in the QH region.

\item {} 
\sphinxAtStartPar
\sphinxstyleliteralstrong{\sphinxupquote{mu\_qh}} (\sphinxstyleliteralemphasis{\sphinxupquote{float}}) \textendash{} Chemical potential in the QH region.

\item {} 
\sphinxAtStartPar
\sphinxstyleliteralstrong{\sphinxupquote{nu}} (\sphinxstyleliteralemphasis{\sphinxupquote{float}}) \textendash{} Filling factor.

\end{itemize}

\item[{Returns}] \leavevmode
\sphinxAtStartPar
The value of hole\sphinxhyphen{}like wave function in QH region.

\item[{Return type}] \leavevmode
\sphinxAtStartPar
float

\end{description}\end{quote}

\end{fulllineitems}

\index{chi\_p() (in module modules.functions)@\spxentry{chi\_p()}\spxextra{in module modules.functions}}

\begin{fulllineitems}
\phantomsection\label{\detokenize{modules:modules.functions.chi_p}}
\pysigstartsignatures
\pysiglinewithargsret{\sphinxcode{\sphinxupquote{modules.functions.}}\sphinxbfcode{\sphinxupquote{chi\_p}}}{\emph{\DUrole{n}{x}}, \emph{\DUrole{n}{E}}, \emph{\DUrole{n}{k}}, \emph{\DUrole{n}{m\_qh}}, \emph{\DUrole{n}{mu\_qh}}, \emph{\DUrole{n}{nu}}}{}
\pysigstopsignatures
\sphinxAtStartPar
Electron\sphinxhyphen{}like wave function in QH region.
\begin{quote}\begin{description}
\item[{Parameters}] \leavevmode\begin{itemize}
\item {} 
\sphinxAtStartPar
\sphinxstyleliteralstrong{\sphinxupquote{x}} (\sphinxstyleliteralemphasis{\sphinxupquote{float}}) \textendash{} x\sphinxhyphen{}coordinate.

\item {} 
\sphinxAtStartPar
\sphinxstyleliteralstrong{\sphinxupquote{E}} (\sphinxstyleliteralemphasis{\sphinxupquote{float}}) \textendash{} Energy measured from the Fermi level.

\item {} 
\sphinxAtStartPar
\sphinxstyleliteralstrong{\sphinxupquote{k}} (\sphinxstyleliteralemphasis{\sphinxupquote{float}}) \textendash{} Momentum along the QH\sphinxhyphen{}SC interface.

\item {} 
\sphinxAtStartPar
\sphinxstyleliteralstrong{\sphinxupquote{m\_qh}} (\sphinxstyleliteralemphasis{\sphinxupquote{float}}) \textendash{} Effective mass in the QH region.

\item {} 
\sphinxAtStartPar
\sphinxstyleliteralstrong{\sphinxupquote{mu\_qh}} (\sphinxstyleliteralemphasis{\sphinxupquote{float}}) \textendash{} Chemical potential in the QH region.

\item {} 
\sphinxAtStartPar
\sphinxstyleliteralstrong{\sphinxupquote{nu}} (\sphinxstyleliteralemphasis{\sphinxupquote{float}}) \textendash{} Filling factor.

\end{itemize}

\item[{Returns}] \leavevmode
\sphinxAtStartPar
The value of electron\sphinxhyphen{}like wave function in QH region.

\item[{Return type}] \leavevmode
\sphinxAtStartPar
float

\end{description}\end{quote}

\end{fulllineitems}

\index{effective\_tau() (in module modules.functions)@\spxentry{effective\_tau()}\spxextra{in module modules.functions}}

\begin{fulllineitems}
\phantomsection\label{\detokenize{modules:modules.functions.effective_tau}}
\pysigstartsignatures
\pysiglinewithargsret{\sphinxcode{\sphinxupquote{modules.functions.}}\sphinxbfcode{\sphinxupquote{effective\_tau}}}{\emph{\DUrole{n}{tau\_0}}, \emph{\DUrole{n}{L\_b}}, \emph{\DUrole{n}{v\_b}}, \emph{\DUrole{n}{mu\_b}}, \emph{\DUrole{n}{delta\_b}}, \emph{\DUrole{n}{phi\_b}}}{}
\pysigstopsignatures
\sphinxAtStartPar
Compute the effective conversion probability at a QH\sphinxhyphen{}SS corner.

\sphinxAtStartPar
The value tau\_0 is the one obtained from the microscopic model while the parameters
labelled with ‘\_b’ correspond to the effective barrier.
\begin{quote}\begin{description}
\item[{Parameters}] \leavevmode\begin{itemize}
\item {} 
\sphinxAtStartPar
\sphinxstyleliteralstrong{\sphinxupquote{tau\_0}} (\sphinxstyleliteralemphasis{\sphinxupquote{str}}) \textendash{} Hole probability computed with the microscopic model.

\item {} 
\sphinxAtStartPar
\sphinxstyleliteralstrong{\sphinxupquote{L\_b}} (\sphinxstyleliteralemphasis{\sphinxupquote{str}}) \textendash{} Length of the barrier.

\item {} 
\sphinxAtStartPar
\sphinxstyleliteralstrong{\sphinxupquote{v\_b}} (\sphinxstyleliteralemphasis{\sphinxupquote{str}}) \textendash{} Velocity in the barrier.

\item {} 
\sphinxAtStartPar
\sphinxstyleliteralstrong{\sphinxupquote{mu\_b}} (\sphinxstyleliteralemphasis{\sphinxupquote{str}}) \textendash{} Chemical potential in the barrier.

\item {} 
\sphinxAtStartPar
\sphinxstyleliteralstrong{\sphinxupquote{delta\_b}} (\sphinxstyleliteralemphasis{\sphinxupquote{str}}) \textendash{} Superconducting gap of the barrier.

\item {} 
\sphinxAtStartPar
\sphinxstyleliteralstrong{\sphinxupquote{phi\_b}} (\sphinxstyleliteralemphasis{\sphinxupquote{str}}) \textendash{} Superconducting phase of the barrier.

\end{itemize}

\end{description}\end{quote}

\end{fulllineitems}

\index{fermi\_momenta() (in module modules.functions)@\spxentry{fermi\_momenta()}\spxextra{in module modules.functions}}

\begin{fulllineitems}
\phantomsection\label{\detokenize{modules:modules.functions.fermi_momenta}}
\pysigstartsignatures
\pysiglinewithargsret{\sphinxcode{\sphinxupquote{modules.functions.}}\sphinxbfcode{\sphinxupquote{fermi\_momenta}}}{\emph{\DUrole{n}{m\_qh}}, \emph{\DUrole{n}{m\_sc}}, \emph{\DUrole{n}{mu\_qh}}, \emph{\DUrole{n}{mu\_sc}}, \emph{\DUrole{n}{nu}}, \emph{\DUrole{n}{delta}}, \emph{\DUrole{n}{V\_barrier}}}{}
\pysigstopsignatures
\sphinxAtStartPar
Positive momentum solutions of the secular equation f(E, k) = 0 at the Fermi level (i.e. at E = 0).
\begin{quote}\begin{description}
\item[{Parameters}] \leavevmode\begin{itemize}
\item {} 
\sphinxAtStartPar
\sphinxstyleliteralstrong{\sphinxupquote{m\_qh}} (\sphinxstyleliteralemphasis{\sphinxupquote{float}}) \textendash{} Effective mass in the QH region.

\item {} 
\sphinxAtStartPar
\sphinxstyleliteralstrong{\sphinxupquote{m\_sc}} (\sphinxstyleliteralemphasis{\sphinxupquote{float}}) \textendash{} Effective mass in the SC region.

\item {} 
\sphinxAtStartPar
\sphinxstyleliteralstrong{\sphinxupquote{mu\_qh}} (\sphinxstyleliteralemphasis{\sphinxupquote{float}}) \textendash{} Chemical potential in the QH region.

\item {} 
\sphinxAtStartPar
\sphinxstyleliteralstrong{\sphinxupquote{mu\_sc}} (\sphinxstyleliteralemphasis{\sphinxupquote{float}}) \textendash{} Chemical potential in the SC region.

\item {} 
\sphinxAtStartPar
\sphinxstyleliteralstrong{\sphinxupquote{nu}} (\sphinxstyleliteralemphasis{\sphinxupquote{float}}) \textendash{} Filling factor.

\item {} 
\sphinxAtStartPar
\sphinxstyleliteralstrong{\sphinxupquote{delta}} (\sphinxstyleliteralemphasis{\sphinxupquote{float}}) \textendash{} Superconducting gap.

\item {} 
\sphinxAtStartPar
\sphinxstyleliteralstrong{\sphinxupquote{V\_barrier}} (\sphinxstyleliteralemphasis{\sphinxupquote{float}}) \textendash{} Height of the delta\sphinxhyphen{}potential barrier.

\end{itemize}

\item[{Returns}] \leavevmode
\sphinxAtStartPar
The positive solutions of the equation f(E=0, k) = 0.

\item[{Return type}] \leavevmode
\sphinxAtStartPar
list

\end{description}\end{quote}

\end{fulllineitems}

\index{hole\_probability() (in module modules.functions)@\spxentry{hole\_probability()}\spxextra{in module modules.functions}}

\begin{fulllineitems}
\phantomsection\label{\detokenize{modules:modules.functions.hole_probability}}
\pysigstartsignatures
\pysiglinewithargsret{\sphinxcode{\sphinxupquote{modules.functions.}}\sphinxbfcode{\sphinxupquote{hole\_probability}}}{\emph{\DUrole{n}{m\_qh}}, \emph{\DUrole{n}{m\_sc}}, \emph{\DUrole{n}{mu\_qh}}, \emph{\DUrole{n}{mu\_sc}}, \emph{\DUrole{n}{nu}}, \emph{\DUrole{n}{delta}}, \emph{\DUrole{n}{V\_barrier}}}{}
\pysigstopsignatures
\sphinxAtStartPar
Compute the hole content f\_h\textasciicircum{}+ at the quasi\sphinxhyphen{}electron crossing, i.e., at k = \sphinxhyphen{}k0.
\begin{quote}\begin{description}
\item[{Parameters}] \leavevmode\begin{itemize}
\item {} 
\sphinxAtStartPar
\sphinxstyleliteralstrong{\sphinxupquote{m\_qh}} (\sphinxstyleliteralemphasis{\sphinxupquote{float}}) \textendash{} Effective mass in the QH region.

\item {} 
\sphinxAtStartPar
\sphinxstyleliteralstrong{\sphinxupquote{m\_sc}} (\sphinxstyleliteralemphasis{\sphinxupquote{float}}) \textendash{} Effective mass in the SC region.

\item {} 
\sphinxAtStartPar
\sphinxstyleliteralstrong{\sphinxupquote{mu\_qh}} (\sphinxstyleliteralemphasis{\sphinxupquote{float}}) \textendash{} Chemical potential in the QH region.

\item {} 
\sphinxAtStartPar
\sphinxstyleliteralstrong{\sphinxupquote{mu\_sc}} (\sphinxstyleliteralemphasis{\sphinxupquote{float}}) \textendash{} Chemical potential in the SC region.

\item {} 
\sphinxAtStartPar
\sphinxstyleliteralstrong{\sphinxupquote{nu}} (\sphinxstyleliteralemphasis{\sphinxupquote{float}}) \textendash{} Filling factor.

\item {} 
\sphinxAtStartPar
\sphinxstyleliteralstrong{\sphinxupquote{delta}} (\sphinxstyleliteralemphasis{\sphinxupquote{float}}) \textendash{} Superconducting gap.

\item {} 
\sphinxAtStartPar
\sphinxstyleliteralstrong{\sphinxupquote{V\_barrier}} (\sphinxstyleliteralemphasis{\sphinxupquote{float}}) \textendash{} Height of the delta\sphinxhyphen{}potential barrier.

\end{itemize}

\item[{Returns}] \leavevmode
\sphinxAtStartPar
Hole probability at the quasi\sphinxhyphen{}electron crossing.

\item[{Return type}] \leavevmode
\sphinxAtStartPar
float

\end{description}\end{quote}

\end{fulllineitems}

\index{hopping() (in module modules.functions)@\spxentry{hopping()}\spxextra{in module modules.functions}}

\begin{fulllineitems}
\phantomsection\label{\detokenize{modules:modules.functions.hopping}}
\pysigstartsignatures
\pysiglinewithargsret{\sphinxcode{\sphinxupquote{modules.functions.}}\sphinxbfcode{\sphinxupquote{hopping}}}{\emph{\DUrole{n}{site1}}, \emph{\DUrole{n}{site2}}, \emph{\DUrole{n}{a}}, \emph{\DUrole{n}{t}}, \emph{\DUrole{n}{mu\_qh}}, \emph{\DUrole{n}{nu}}}{}
\pysigstopsignatures
\sphinxAtStartPar
Define hopping energies in QH and SC regions.
\begin{quote}\begin{description}
\item[{Parameters}] \leavevmode\begin{itemize}
\item {} 
\sphinxAtStartPar
\sphinxstyleliteralstrong{\sphinxupquote{site1}} \textendash{} Kwant site.

\item {} 
\sphinxAtStartPar
\sphinxstyleliteralstrong{\sphinxupquote{site2}} \textendash{} Kwant site.

\item {} 
\sphinxAtStartPar
\sphinxstyleliteralstrong{\sphinxupquote{a}} (\sphinxstyleliteralemphasis{\sphinxupquote{float}}) \textendash{} Lattice spacing.

\item {} 
\sphinxAtStartPar
\sphinxstyleliteralstrong{\sphinxupquote{t}} (\sphinxstyleliteralemphasis{\sphinxupquote{float}}) \textendash{} Hopping energy at zero field.

\item {} 
\sphinxAtStartPar
\sphinxstyleliteralstrong{\sphinxupquote{mu\_qh}} (\sphinxstyleliteralemphasis{\sphinxupquote{float}}) \textendash{} Chemical potential in the QH region.

\item {} 
\sphinxAtStartPar
\sphinxstyleliteralstrong{\sphinxupquote{nu}} (\sphinxstyleliteralemphasis{\sphinxupquote{float}}) \textendash{} Filling factor.

\end{itemize}

\item[{Returns}] \leavevmode
\sphinxAtStartPar
Hopping energy.

\item[{Return type}] \leavevmode
\sphinxAtStartPar
float

\end{description}\end{quote}

\end{fulllineitems}

\index{hopping\_qh() (in module modules.functions)@\spxentry{hopping\_qh()}\spxextra{in module modules.functions}}

\begin{fulllineitems}
\phantomsection\label{\detokenize{modules:modules.functions.hopping_qh}}
\pysigstartsignatures
\pysiglinewithargsret{\sphinxcode{\sphinxupquote{modules.functions.}}\sphinxbfcode{\sphinxupquote{hopping\_qh}}}{\emph{\DUrole{n}{site1}}, \emph{\DUrole{n}{site2}}, \emph{\DUrole{n}{a}}, \emph{\DUrole{n}{t}}, \emph{\DUrole{n}{mu\_qh}}, \emph{\DUrole{n}{nu}}}{}
\pysigstopsignatures
\sphinxAtStartPar
Define hopping energy in QH region.
\begin{quote}\begin{description}
\item[{Parameters}] \leavevmode\begin{itemize}
\item {} 
\sphinxAtStartPar
\sphinxstyleliteralstrong{\sphinxupquote{site1}} \textendash{} Kwant site.

\item {} 
\sphinxAtStartPar
\sphinxstyleliteralstrong{\sphinxupquote{site2}} \textendash{} Kwant site.

\item {} 
\sphinxAtStartPar
\sphinxstyleliteralstrong{\sphinxupquote{a}} (\sphinxstyleliteralemphasis{\sphinxupquote{float}}) \textendash{} Lattice spacing.

\item {} 
\sphinxAtStartPar
\sphinxstyleliteralstrong{\sphinxupquote{t}} (\sphinxstyleliteralemphasis{\sphinxupquote{float}}) \textendash{} Hopping energy at zero field.

\item {} 
\sphinxAtStartPar
\sphinxstyleliteralstrong{\sphinxupquote{mu\_qh}} (\sphinxstyleliteralemphasis{\sphinxupquote{float}}) \textendash{} Chemical potential in the QH region.

\item {} 
\sphinxAtStartPar
\sphinxstyleliteralstrong{\sphinxupquote{nu}} (\sphinxstyleliteralemphasis{\sphinxupquote{float}}) \textendash{} Filling factor.

\end{itemize}

\item[{Returns}] \leavevmode
\sphinxAtStartPar
Hopping energy.

\item[{Return type}] \leavevmode
\sphinxAtStartPar
float

\end{description}\end{quote}

\end{fulllineitems}

\index{hopping\_sc() (in module modules.functions)@\spxentry{hopping\_sc()}\spxextra{in module modules.functions}}

\begin{fulllineitems}
\phantomsection\label{\detokenize{modules:modules.functions.hopping_sc}}
\pysigstartsignatures
\pysiglinewithargsret{\sphinxcode{\sphinxupquote{modules.functions.}}\sphinxbfcode{\sphinxupquote{hopping\_sc}}}{\emph{\DUrole{n}{site1}}, \emph{\DUrole{n}{site2}}, \emph{\DUrole{n}{t}}}{}
\pysigstopsignatures
\sphinxAtStartPar
Define hopping energy in SC region.
\begin{quote}\begin{description}
\item[{Parameters}] \leavevmode\begin{itemize}
\item {} 
\sphinxAtStartPar
\sphinxstyleliteralstrong{\sphinxupquote{site1}} \textendash{} Kwant site.

\item {} 
\sphinxAtStartPar
\sphinxstyleliteralstrong{\sphinxupquote{site2}} \textendash{} Kwant site.

\item {} 
\sphinxAtStartPar
\sphinxstyleliteralstrong{\sphinxupquote{t}} (\sphinxstyleliteralemphasis{\sphinxupquote{float}}) \textendash{} Hopping energy at zero field.

\end{itemize}

\item[{Returns}] \leavevmode
\sphinxAtStartPar
Hopping energy.

\item[{Return type}] \leavevmode
\sphinxAtStartPar
float

\end{description}\end{quote}

\end{fulllineitems}

\index{kronecker\_delta() (in module modules.functions)@\spxentry{kronecker\_delta()}\spxextra{in module modules.functions}}

\begin{fulllineitems}
\phantomsection\label{\detokenize{modules:modules.functions.kronecker_delta}}
\pysigstartsignatures
\pysiglinewithargsret{\sphinxcode{\sphinxupquote{modules.functions.}}\sphinxbfcode{\sphinxupquote{kronecker\_delta}}}{\emph{\DUrole{n}{x}}}{}
\pysigstopsignatures
\sphinxAtStartPar
Kronecker delta function.
\begin{quote}\begin{description}
\item[{Parameters}] \leavevmode
\sphinxAtStartPar
\sphinxstyleliteralstrong{\sphinxupquote{x}} (\sphinxstyleliteralemphasis{\sphinxupquote{float}}) \textendash{} x\sphinxhyphen{}coordinate.

\item[{Returns}] \leavevmode
\sphinxAtStartPar
\begin{enumerate}
\sphinxsetlistlabels{\arabic}{enumi}{enumii}{}{.}%
\item {} 
\sphinxAtStartPar
if x=0 and 0. otherwise.

\end{enumerate}


\item[{Return type}] \leavevmode
\sphinxAtStartPar
float

\end{description}\end{quote}

\end{fulllineitems}

\index{onsite() (in module modules.functions)@\spxentry{onsite()}\spxextra{in module modules.functions}}

\begin{fulllineitems}
\phantomsection\label{\detokenize{modules:modules.functions.onsite}}
\pysigstartsignatures
\pysiglinewithargsret{\sphinxcode{\sphinxupquote{modules.functions.}}\sphinxbfcode{\sphinxupquote{onsite}}}{\emph{\DUrole{n}{site}}, \emph{\DUrole{n}{a}}, \emph{\DUrole{n}{t}}, \emph{\DUrole{n}{mu\_qh}}, \emph{\DUrole{n}{mu\_sc}}, \emph{\DUrole{n}{delta}}, \emph{\DUrole{n}{Z}}}{}
\pysigstopsignatures
\sphinxAtStartPar
Define onsite energies in QH and SC regions including a delta\sphinxhyphen{}potential barrier.
\begin{quote}\begin{description}
\item[{Parameters}] \leavevmode\begin{itemize}
\item {} 
\sphinxAtStartPar
\sphinxstyleliteralstrong{\sphinxupquote{site}} \textendash{} Kwant site.

\item {} 
\sphinxAtStartPar
\sphinxstyleliteralstrong{\sphinxupquote{a}} (\sphinxstyleliteralemphasis{\sphinxupquote{float}}) \textendash{} Lattice spacing.

\item {} 
\sphinxAtStartPar
\sphinxstyleliteralstrong{\sphinxupquote{t}} (\sphinxstyleliteralemphasis{\sphinxupquote{float}}) \textendash{} Hopping energy at zero field.

\item {} 
\sphinxAtStartPar
\sphinxstyleliteralstrong{\sphinxupquote{mu\_qh}} (\sphinxstyleliteralemphasis{\sphinxupquote{float}}) \textendash{} Chemical potential in the QH region.

\item {} 
\sphinxAtStartPar
\sphinxstyleliteralstrong{\sphinxupquote{mu\_sc}} (\sphinxstyleliteralemphasis{\sphinxupquote{float}}) \textendash{} Chemical potential in the SC region.

\item {} 
\sphinxAtStartPar
\sphinxstyleliteralstrong{\sphinxupquote{delta}} (\sphinxstyleliteralemphasis{\sphinxupquote{float}}) \textendash{} Superconducting gap.

\item {} 
\sphinxAtStartPar
\sphinxstyleliteralstrong{\sphinxupquote{Z}} (\sphinxstyleliteralemphasis{\sphinxupquote{float}}) \textendash{} Barrier strength.

\end{itemize}

\item[{Returns}] \leavevmode
\sphinxAtStartPar
Onsite energy.

\item[{Return type}] \leavevmode
\sphinxAtStartPar
float

\end{description}\end{quote}

\end{fulllineitems}

\index{onsite\_qh() (in module modules.functions)@\spxentry{onsite\_qh()}\spxextra{in module modules.functions}}

\begin{fulllineitems}
\phantomsection\label{\detokenize{modules:modules.functions.onsite_qh}}
\pysigstartsignatures
\pysiglinewithargsret{\sphinxcode{\sphinxupquote{modules.functions.}}\sphinxbfcode{\sphinxupquote{onsite\_qh}}}{\emph{\DUrole{n}{site}}, \emph{\DUrole{n}{t}}, \emph{\DUrole{n}{mu\_qh}}}{}
\pysigstopsignatures
\sphinxAtStartPar
Define onsite energy in QH region.
\begin{quote}\begin{description}
\item[{Parameters}] \leavevmode\begin{itemize}
\item {} 
\sphinxAtStartPar
\sphinxstyleliteralstrong{\sphinxupquote{site}} \textendash{} Kwant site.

\item {} 
\sphinxAtStartPar
\sphinxstyleliteralstrong{\sphinxupquote{t}} (\sphinxstyleliteralemphasis{\sphinxupquote{float}}) \textendash{} Hopping energy at zero field.

\item {} 
\sphinxAtStartPar
\sphinxstyleliteralstrong{\sphinxupquote{mu\_qh}} (\sphinxstyleliteralemphasis{\sphinxupquote{float}}) \textendash{} Chemical potential in the QH region.

\end{itemize}

\item[{Returns}] \leavevmode
\sphinxAtStartPar
Onsite energy.

\item[{Return type}] \leavevmode
\sphinxAtStartPar
float

\end{description}\end{quote}

\end{fulllineitems}

\index{onsite\_sc() (in module modules.functions)@\spxentry{onsite\_sc()}\spxextra{in module modules.functions}}

\begin{fulllineitems}
\phantomsection\label{\detokenize{modules:modules.functions.onsite_sc}}
\pysigstartsignatures
\pysiglinewithargsret{\sphinxcode{\sphinxupquote{modules.functions.}}\sphinxbfcode{\sphinxupquote{onsite\_sc}}}{\emph{\DUrole{n}{site}}, \emph{\DUrole{n}{t}}, \emph{\DUrole{n}{mu\_sc}}, \emph{\DUrole{n}{delta}}}{}
\pysigstopsignatures
\sphinxAtStartPar
Define onsite energy in SC region.
\begin{quote}\begin{description}
\item[{Parameters}] \leavevmode\begin{itemize}
\item {} 
\sphinxAtStartPar
\sphinxstyleliteralstrong{\sphinxupquote{site}} \textendash{} Kwant site.

\item {} 
\sphinxAtStartPar
\sphinxstyleliteralstrong{\sphinxupquote{t}} (\sphinxstyleliteralemphasis{\sphinxupquote{float}}) \textendash{} Hopping energy at zero field.

\item {} 
\sphinxAtStartPar
\sphinxstyleliteralstrong{\sphinxupquote{mu\_sc}} (\sphinxstyleliteralemphasis{\sphinxupquote{float}}) \textendash{} Chemical potential in the SC region.

\item {} 
\sphinxAtStartPar
\sphinxstyleliteralstrong{\sphinxupquote{delta}} (\sphinxstyleliteralemphasis{\sphinxupquote{float}}) \textendash{} Superconducting gap.

\end{itemize}

\item[{Returns}] \leavevmode
\sphinxAtStartPar
Onsite energy.

\item[{Return type}] \leavevmode
\sphinxAtStartPar
float

\end{description}\end{quote}

\end{fulllineitems}

\index{phi\_m() (in module modules.functions)@\spxentry{phi\_m()}\spxextra{in module modules.functions}}

\begin{fulllineitems}
\phantomsection\label{\detokenize{modules:modules.functions.phi_m}}
\pysigstartsignatures
\pysiglinewithargsret{\sphinxcode{\sphinxupquote{modules.functions.}}\sphinxbfcode{\sphinxupquote{phi\_m}}}{\emph{\DUrole{n}{x}}, \emph{\DUrole{n}{E}}, \emph{\DUrole{n}{k}}, \emph{\DUrole{n}{m\_sc}}, \emph{\DUrole{n}{mu\_sc}}, \emph{\DUrole{n}{delta}}}{}
\pysigstopsignatures
\sphinxAtStartPar
Hole\sphinxhyphen{}like wave function in SC region.
\begin{quote}\begin{description}
\item[{Parameters}] \leavevmode\begin{itemize}
\item {} 
\sphinxAtStartPar
\sphinxstyleliteralstrong{\sphinxupquote{x}} (\sphinxstyleliteralemphasis{\sphinxupquote{float}}) \textendash{} x\sphinxhyphen{}coordinate.

\item {} 
\sphinxAtStartPar
\sphinxstyleliteralstrong{\sphinxupquote{E}} (\sphinxstyleliteralemphasis{\sphinxupquote{float}}) \textendash{} Energy measured from the Fermi level.

\item {} 
\sphinxAtStartPar
\sphinxstyleliteralstrong{\sphinxupquote{k}} (\sphinxstyleliteralemphasis{\sphinxupquote{float}}) \textendash{} Momentum along the QH\sphinxhyphen{}SC interface.

\item {} 
\sphinxAtStartPar
\sphinxstyleliteralstrong{\sphinxupquote{m\_sc}} (\sphinxstyleliteralemphasis{\sphinxupquote{float}}) \textendash{} Effective mass in the SC region.

\item {} 
\sphinxAtStartPar
\sphinxstyleliteralstrong{\sphinxupquote{mu\_sc}} (\sphinxstyleliteralemphasis{\sphinxupquote{float}}) \textendash{} Chemical potential in the SC region.

\item {} 
\sphinxAtStartPar
\sphinxstyleliteralstrong{\sphinxupquote{delta}} (\sphinxstyleliteralemphasis{\sphinxupquote{float}}) \textendash{} Superconducting gap.

\end{itemize}

\item[{Returns}] \leavevmode
\sphinxAtStartPar
The value of hole\sphinxhyphen{}like wave function in SC region.

\item[{Return type}] \leavevmode
\sphinxAtStartPar
float

\end{description}\end{quote}

\end{fulllineitems}

\index{phi\_p() (in module modules.functions)@\spxentry{phi\_p()}\spxextra{in module modules.functions}}

\begin{fulllineitems}
\phantomsection\label{\detokenize{modules:modules.functions.phi_p}}
\pysigstartsignatures
\pysiglinewithargsret{\sphinxcode{\sphinxupquote{modules.functions.}}\sphinxbfcode{\sphinxupquote{phi\_p}}}{\emph{\DUrole{n}{x}}, \emph{\DUrole{n}{E}}, \emph{\DUrole{n}{k}}, \emph{\DUrole{n}{m\_sc}}, \emph{\DUrole{n}{mu\_sc}}, \emph{\DUrole{n}{delta}}}{}
\pysigstopsignatures
\sphinxAtStartPar
Electron\sphinxhyphen{}like wave function in SC region.
\begin{quote}\begin{description}
\item[{Parameters}] \leavevmode\begin{itemize}
\item {} 
\sphinxAtStartPar
\sphinxstyleliteralstrong{\sphinxupquote{x}} (\sphinxstyleliteralemphasis{\sphinxupquote{float}}) \textendash{} x\sphinxhyphen{}coordinate.

\item {} 
\sphinxAtStartPar
\sphinxstyleliteralstrong{\sphinxupquote{E}} (\sphinxstyleliteralemphasis{\sphinxupquote{float}}) \textendash{} Energy measured from the Fermi level.

\item {} 
\sphinxAtStartPar
\sphinxstyleliteralstrong{\sphinxupquote{k}} (\sphinxstyleliteralemphasis{\sphinxupquote{float}}) \textendash{} Momentum along the QH\sphinxhyphen{}SC interface.

\item {} 
\sphinxAtStartPar
\sphinxstyleliteralstrong{\sphinxupquote{m\_sc}} (\sphinxstyleliteralemphasis{\sphinxupquote{float}}) \textendash{} Effective mass in the SC region.

\item {} 
\sphinxAtStartPar
\sphinxstyleliteralstrong{\sphinxupquote{mu\_sc}} (\sphinxstyleliteralemphasis{\sphinxupquote{float}}) \textendash{} Chemical potential in the SC region.

\item {} 
\sphinxAtStartPar
\sphinxstyleliteralstrong{\sphinxupquote{delta}} (\sphinxstyleliteralemphasis{\sphinxupquote{float}}) \textendash{} Superconducting gap.

\end{itemize}

\item[{Returns}] \leavevmode
\sphinxAtStartPar
The value of electron\sphinxhyphen{}like wave function in SC region.

\item[{Return type}] \leavevmode
\sphinxAtStartPar
float

\end{description}\end{quote}

\end{fulllineitems}

\index{secular\_equation() (in module modules.functions)@\spxentry{secular\_equation()}\spxextra{in module modules.functions}}

\begin{fulllineitems}
\phantomsection\label{\detokenize{modules:modules.functions.secular_equation}}
\pysigstartsignatures
\pysiglinewithargsret{\sphinxcode{\sphinxupquote{modules.functions.}}\sphinxbfcode{\sphinxupquote{secular\_equation}}}{\emph{\DUrole{n}{k}}, \emph{\DUrole{n}{E}}, \emph{\DUrole{n}{m\_qh}}, \emph{\DUrole{n}{m\_sc}}, \emph{\DUrole{n}{mu\_qh}}, \emph{\DUrole{n}{mu\_sc}}, \emph{\DUrole{n}{omega}}, \emph{\DUrole{n}{delta}}, \emph{\DUrole{n}{V\_barrier}}}{}
\pysigstopsignatures
\sphinxAtStartPar
Value of f(k, E) used to compute the energy spectrum of the CAES
and the Fermi momenta by solving the secular equation f(k, E) = 0.
\begin{quote}\begin{description}
\item[{Parameters}] \leavevmode\begin{itemize}
\item {} 
\sphinxAtStartPar
\sphinxstyleliteralstrong{\sphinxupquote{E}} (\sphinxstyleliteralemphasis{\sphinxupquote{float}}) \textendash{} Energy measured from the Fermi level.

\item {} 
\sphinxAtStartPar
\sphinxstyleliteralstrong{\sphinxupquote{k}} (\sphinxstyleliteralemphasis{\sphinxupquote{float}}) \textendash{} Momentum along the QH\sphinxhyphen{}SC interface.

\item {} 
\sphinxAtStartPar
\sphinxstyleliteralstrong{\sphinxupquote{m\_qh}} (\sphinxstyleliteralemphasis{\sphinxupquote{float}}) \textendash{} Effective mass in the QH region.

\item {} 
\sphinxAtStartPar
\sphinxstyleliteralstrong{\sphinxupquote{m\_sc}} (\sphinxstyleliteralemphasis{\sphinxupquote{float}}) \textendash{} Effective mass in the SC region.

\item {} 
\sphinxAtStartPar
\sphinxstyleliteralstrong{\sphinxupquote{mu\_qh}} (\sphinxstyleliteralemphasis{\sphinxupquote{float}}) \textendash{} Chemical potential in the QH region.

\item {} 
\sphinxAtStartPar
\sphinxstyleliteralstrong{\sphinxupquote{mu\_sc}} (\sphinxstyleliteralemphasis{\sphinxupquote{float}}) \textendash{} Chemical potential in the SC region.

\item {} 
\sphinxAtStartPar
\sphinxstyleliteralstrong{\sphinxupquote{omega}} (\sphinxstyleliteralemphasis{\sphinxupquote{float}}) \textendash{} Cyclotron frequency.

\item {} 
\sphinxAtStartPar
\sphinxstyleliteralstrong{\sphinxupquote{delta}} (\sphinxstyleliteralemphasis{\sphinxupquote{float}}) \textendash{} Superconducting gap.

\item {} 
\sphinxAtStartPar
\sphinxstyleliteralstrong{\sphinxupquote{V\_barrier}} (\sphinxstyleliteralemphasis{\sphinxupquote{float}}) \textendash{} Height of the delta\sphinxhyphen{}potential barrier.

\end{itemize}

\item[{Returns}] \leavevmode
\sphinxAtStartPar
The value of f(k, E).

\item[{Return type}] \leavevmode
\sphinxAtStartPar
float

\end{description}\end{quote}

\end{fulllineitems}

\index{velocity() (in module modules.functions)@\spxentry{velocity()}\spxextra{in module modules.functions}}

\begin{fulllineitems}
\phantomsection\label{\detokenize{modules:modules.functions.velocity}}
\pysigstartsignatures
\pysiglinewithargsret{\sphinxcode{\sphinxupquote{modules.functions.}}\sphinxbfcode{\sphinxupquote{velocity}}}{\emph{\DUrole{n}{m\_qh}}, \emph{\DUrole{n}{m\_sc}}, \emph{\DUrole{n}{mu\_qh}}, \emph{\DUrole{n}{mu\_sc}}, \emph{\DUrole{n}{nu}}, \emph{\DUrole{n}{delta}}, \emph{\DUrole{n}{V\_barrier}}}{}
\pysigstopsignatures
\sphinxAtStartPar
Compute the velocity of the CAES.
\begin{quote}\begin{description}
\item[{Parameters}] \leavevmode\begin{itemize}
\item {} 
\sphinxAtStartPar
\sphinxstyleliteralstrong{\sphinxupquote{m\_qh}} (\sphinxstyleliteralemphasis{\sphinxupquote{float}}) \textendash{} Effective mass in the QH region.

\item {} 
\sphinxAtStartPar
\sphinxstyleliteralstrong{\sphinxupquote{m\_sc}} (\sphinxstyleliteralemphasis{\sphinxupquote{float}}) \textendash{} Effective mass in the SC region.

\item {} 
\sphinxAtStartPar
\sphinxstyleliteralstrong{\sphinxupquote{mu\_qh}} (\sphinxstyleliteralemphasis{\sphinxupquote{float}}) \textendash{} Chemical potential in the QH region.

\item {} 
\sphinxAtStartPar
\sphinxstyleliteralstrong{\sphinxupquote{mu\_sc}} (\sphinxstyleliteralemphasis{\sphinxupquote{float}}) \textendash{} Chemical potential in the SC region.

\item {} 
\sphinxAtStartPar
\sphinxstyleliteralstrong{\sphinxupquote{nu}} (\sphinxstyleliteralemphasis{\sphinxupquote{float}}) \textendash{} Filling factor.

\item {} 
\sphinxAtStartPar
\sphinxstyleliteralstrong{\sphinxupquote{delta}} (\sphinxstyleliteralemphasis{\sphinxupquote{float}}) \textendash{} Superconducting gap.

\item {} 
\sphinxAtStartPar
\sphinxstyleliteralstrong{\sphinxupquote{V\_barrier}} (\sphinxstyleliteralemphasis{\sphinxupquote{float}}) \textendash{} Height of the delta\sphinxhyphen{}potential barrier.

\end{itemize}

\item[{Returns}] \leavevmode
\sphinxAtStartPar
The value of the velocity.

\item[{Return type}] \leavevmode
\sphinxAtStartPar
float

\end{description}\end{quote}

\end{fulllineitems}



\section{system.py}
\label{\detokenize{modules:module-modules.system}}\label{\detokenize{modules:system-py}}\index{module@\spxentry{module}!modules.system@\spxentry{modules.system}}\index{modules.system@\spxentry{modules.system}!module@\spxentry{module}}
\sphinxAtStartPar
Classes constructing different Kwant’s systems.
\index{Device (class in modules.system)@\spxentry{Device}\spxextra{class in modules.system}}

\begin{fulllineitems}
\phantomsection\label{\detokenize{modules:modules.system.Device}}
\pysigstartsignatures
\pysiglinewithargsret{\sphinxbfcode{\sphinxupquote{class\DUrole{w}{  }}}\sphinxcode{\sphinxupquote{modules.system.}}\sphinxbfcode{\sphinxupquote{Device}}}{\emph{\DUrole{n}{theta\_1}}, \emph{\DUrole{n}{theta\_2}}, \emph{\DUrole{n}{params}}}{}
\pysigstopsignatures
\sphinxAtStartPar
Construct a QH\sphinxhyphen{}SC junction with arbitrary QH angles and a rectangle\sphinxhyphen{}shaped SC.
\begin{quote}\begin{description}
\item[{Parameters}] \leavevmode\begin{itemize}
\item {} 
\sphinxAtStartPar
\sphinxstyleliteralstrong{\sphinxupquote{theta\_1}} (\sphinxstyleliteralemphasis{\sphinxupquote{float}}) \textendash{} First QH angle (in degrees).

\item {} 
\sphinxAtStartPar
\sphinxstyleliteralstrong{\sphinxupquote{theta\_2}} (\sphinxstyleliteralemphasis{\sphinxupquote{float}}) \textendash{} Second QH angle (in degrees).

\item {} 
\sphinxAtStartPar
\sphinxstyleliteralstrong{\sphinxupquote{params}} (\sphinxstyleliteralemphasis{\sphinxupquote{dict}}) \textendash{} System’s parameters.

\end{itemize}

\end{description}\end{quote}
\index{make\_system() (modules.system.Device method)@\spxentry{make\_system()}\spxextra{modules.system.Device method}}

\begin{fulllineitems}
\phantomsection\label{\detokenize{modules:modules.system.Device.make_system}}
\pysigstartsignatures
\pysiglinewithargsret{\sphinxbfcode{\sphinxupquote{make\_system}}}{\emph{\DUrole{n}{onsite}}, \emph{\DUrole{n}{hopping}}}{}
\pysigstopsignatures
\sphinxAtStartPar
Make the (unfinalized) system.
\begin{quote}\begin{description}
\item[{Parameters}] \leavevmode\begin{itemize}
\item {} 
\sphinxAtStartPar
\sphinxstyleliteralstrong{\sphinxupquote{onsite}} (\sphinxstyleliteralemphasis{\sphinxupquote{fun}}) \textendash{} Onsite energy function.

\item {} 
\sphinxAtStartPar
\sphinxstyleliteralstrong{\sphinxupquote{hopping}} (\sphinxstyleliteralemphasis{\sphinxupquote{fun}}) \textendash{} Hopping energy function.

\end{itemize}

\item[{Returns}] \leavevmode
\sphinxAtStartPar
Unfinalized Kwant system.

\end{description}\end{quote}

\end{fulllineitems}


\end{fulllineitems}

\index{DeviceInfinite (class in modules.system)@\spxentry{DeviceInfinite}\spxextra{class in modules.system}}

\begin{fulllineitems}
\phantomsection\label{\detokenize{modules:modules.system.DeviceInfinite}}
\pysigstartsignatures
\pysiglinewithargsret{\sphinxbfcode{\sphinxupquote{class\DUrole{w}{  }}}\sphinxcode{\sphinxupquote{modules.system.}}\sphinxbfcode{\sphinxupquote{DeviceInfinite}}}{\emph{\DUrole{n}{params}}}{}
\pysigstopsignatures
\sphinxAtStartPar
Construct a an infinite QH\sphinxhyphen{}SC interface (lead).
\begin{quote}\begin{description}
\item[{Parameters}] \leavevmode
\sphinxAtStartPar
\sphinxstyleliteralstrong{\sphinxupquote{params}} (\sphinxstyleliteralemphasis{\sphinxupquote{dict}}) \textendash{} System’s parameters.

\end{description}\end{quote}
\index{make\_system() (modules.system.DeviceInfinite method)@\spxentry{make\_system()}\spxextra{modules.system.DeviceInfinite method}}

\begin{fulllineitems}
\phantomsection\label{\detokenize{modules:modules.system.DeviceInfinite.make_system}}
\pysigstartsignatures
\pysiglinewithargsret{\sphinxbfcode{\sphinxupquote{make\_system}}}{\emph{\DUrole{n}{onsite}}, \emph{\DUrole{n}{hopping}}}{}
\pysigstopsignatures
\sphinxAtStartPar
Make the (unfinalized) lead.
\begin{quote}\begin{description}
\item[{Parameters}] \leavevmode\begin{itemize}
\item {} 
\sphinxAtStartPar
\sphinxstyleliteralstrong{\sphinxupquote{onsite}} (\sphinxstyleliteralemphasis{\sphinxupquote{fun}}) \textendash{} Onsite energy function.

\item {} 
\sphinxAtStartPar
\sphinxstyleliteralstrong{\sphinxupquote{hopping}} (\sphinxstyleliteralemphasis{\sphinxupquote{fun}}) \textendash{} Hopping energy function.

\end{itemize}

\item[{Returns}] \leavevmode
\sphinxAtStartPar
Unfinalized Kwant lead.

\end{description}\end{quote}

\end{fulllineitems}


\end{fulllineitems}

\index{DeviceSingleCorner (class in modules.system)@\spxentry{DeviceSingleCorner}\spxextra{class in modules.system}}

\begin{fulllineitems}
\phantomsection\label{\detokenize{modules:modules.system.DeviceSingleCorner}}
\pysigstartsignatures
\pysiglinewithargsret{\sphinxbfcode{\sphinxupquote{class\DUrole{w}{  }}}\sphinxcode{\sphinxupquote{modules.system.}}\sphinxbfcode{\sphinxupquote{DeviceSingleCorner}}}{\emph{\DUrole{n}{theta\_qh}}, \emph{\DUrole{n}{theta\_sc}}, \emph{\DUrole{n}{params}}, \emph{\DUrole{n}{small}\DUrole{o}{=}\DUrole{default_value}{False}}}{}
\pysigstopsignatures
\sphinxAtStartPar
Construct a semi\sphinxhyphen{}infinite junction with a single corner.
\begin{quote}\begin{description}
\item[{Parameters}] \leavevmode\begin{itemize}
\item {} 
\sphinxAtStartPar
\sphinxstyleliteralstrong{\sphinxupquote{theta\_qh}} (\sphinxstyleliteralemphasis{\sphinxupquote{float}}) \textendash{} QH angle (in degrees).

\item {} 
\sphinxAtStartPar
\sphinxstyleliteralstrong{\sphinxupquote{theta\_sc}} (\sphinxstyleliteralemphasis{\sphinxupquote{str}}) \textendash{} SC angle (in degrees).

\item {} 
\sphinxAtStartPar
\sphinxstyleliteralstrong{\sphinxupquote{params}} (\sphinxstyleliteralemphasis{\sphinxupquote{dict}}) \textendash{} System’s parameters.

\item {} 
\sphinxAtStartPar
\sphinxstyleliteralstrong{\sphinxupquote{small}} (\sphinxstyleliteralemphasis{\sphinxupquote{bool}}) \textendash{} Must be set to True when small dimensions are used,
default to False.

\end{itemize}

\end{description}\end{quote}
\index{make\_system() (modules.system.DeviceSingleCorner method)@\spxentry{make\_system()}\spxextra{modules.system.DeviceSingleCorner method}}

\begin{fulllineitems}
\phantomsection\label{\detokenize{modules:modules.system.DeviceSingleCorner.make_system}}
\pysigstartsignatures
\pysiglinewithargsret{\sphinxbfcode{\sphinxupquote{make\_system}}}{\emph{\DUrole{n}{onsite}}, \emph{\DUrole{n}{onsite\_qh}}, \emph{\DUrole{n}{onsite\_sc}}, \emph{\DUrole{n}{hopping}}, \emph{\DUrole{n}{hopping\_qh}}, \emph{\DUrole{n}{hopping\_sc}}}{}
\pysigstopsignatures
\sphinxAtStartPar
Make the (unfinalized) system.
\begin{quote}\begin{description}
\item[{Parameters}] \leavevmode\begin{itemize}
\item {} 
\sphinxAtStartPar
\sphinxstyleliteralstrong{\sphinxupquote{onsite}} (\sphinxstyleliteralemphasis{\sphinxupquote{fun}}) \textendash{} Onsite energy function.

\item {} 
\sphinxAtStartPar
\sphinxstyleliteralstrong{\sphinxupquote{onsite\_qh}} (\sphinxstyleliteralemphasis{\sphinxupquote{fun}}) \textendash{} Onsite energy function in QH region.

\item {} 
\sphinxAtStartPar
\sphinxstyleliteralstrong{\sphinxupquote{onsite\_sc}} (\sphinxstyleliteralemphasis{\sphinxupquote{fun}}) \textendash{} Onsite energy function in SC region.

\item {} 
\sphinxAtStartPar
\sphinxstyleliteralstrong{\sphinxupquote{hopping}} (\sphinxstyleliteralemphasis{\sphinxupquote{fun}}) \textendash{} Hopping energy function.

\item {} 
\sphinxAtStartPar
\sphinxstyleliteralstrong{\sphinxupquote{hopping\_qh}} (\sphinxstyleliteralemphasis{\sphinxupquote{fun}}) \textendash{} Hopping energy function in QH region.

\item {} 
\sphinxAtStartPar
\sphinxstyleliteralstrong{\sphinxupquote{hopping\_sc}} (\sphinxstyleliteralemphasis{\sphinxupquote{fun}}) \textendash{} Hopping energy function in SC region.

\end{itemize}

\item[{Returns}] \leavevmode
\sphinxAtStartPar
Unfinalized Kwant system.

\end{description}\end{quote}

\end{fulllineitems}


\end{fulllineitems}



\section{utils.py}
\label{\detokenize{modules:module-modules.utils}}\label{\detokenize{modules:utils-py}}\index{module@\spxentry{module}!modules.utils@\spxentry{modules.utils}}\index{modules.utils@\spxentry{modules.utils}!module@\spxentry{module}}
\sphinxAtStartPar
Definitions of the functions used in \sphinxstyleemphasis{calculations.py}.
\index{compute\_corner\_transmissions() (in module modules.utils)@\spxentry{compute\_corner\_transmissions()}\spxextra{in module modules.utils}}

\begin{fulllineitems}
\phantomsection\label{\detokenize{modules:modules.utils.compute_corner_transmissions}}
\pysigstartsignatures
\pysiglinewithargsret{\sphinxcode{\sphinxupquote{modules.utils.}}\sphinxbfcode{\sphinxupquote{compute\_corner\_transmissions}}}{\emph{\DUrole{n}{device}}, \emph{\DUrole{n}{energy}\DUrole{o}{=}\DUrole{default_value}{0.0}}}{}
\pysigstopsignatures
\sphinxAtStartPar
Compute the corner transmission amplitudes with Kwant.
\begin{quote}\begin{description}
\item[{Parameters}] \leavevmode\begin{itemize}
\item {} 
\sphinxAtStartPar
\sphinxstyleliteralstrong{\sphinxupquote{device}} (\sphinxstyleliteralemphasis{\sphinxupquote{object}}) \textendash{} The QH\sphinxhyphen{}SC corner.

\item {} 
\sphinxAtStartPar
\sphinxstyleliteralstrong{\sphinxupquote{energy}} (\sphinxstyleliteralemphasis{\sphinxupquote{float}}) \textendash{} Value of the energy, default to 0.

\end{itemize}

\item[{Returns}] \leavevmode
\sphinxAtStartPar
The normal and Andreev amplitudes : t\_ee, t\_he.

\item[{Return type}] \leavevmode
\sphinxAtStartPar
list

\end{description}\end{quote}

\end{fulllineitems}

\index{compute\_delta\_b\_and\_phi\_b() (in module modules.utils)@\spxentry{compute\_delta\_b\_and\_phi\_b()}\spxextra{in module modules.utils}}

\begin{fulllineitems}
\phantomsection\label{\detokenize{modules:modules.utils.compute_delta_b_and_phi_b}}
\pysigstartsignatures
\pysiglinewithargsret{\sphinxcode{\sphinxupquote{modules.utils.}}\sphinxbfcode{\sphinxupquote{compute\_delta\_b\_and\_phi\_b}}}{\emph{\DUrole{n}{device}}, \emph{\DUrole{n}{L\_b}}, \emph{\DUrole{n}{v\_b}}, \emph{\DUrole{n}{mu\_b}}}{}
\pysigstopsignatures
\sphinxAtStartPar
Compute the effective barrier parameters delta\_b and phi\_b.

\sphinxAtStartPar
Here have to give L\_b, v\_b, and mu\_b as inputs.
\begin{quote}\begin{description}
\item[{Parameters}] \leavevmode\begin{itemize}
\item {} 
\sphinxAtStartPar
\sphinxstyleliteralstrong{\sphinxupquote{device}} (\sphinxstyleliteralemphasis{\sphinxupquote{object}}) \textendash{} The QH\sphinxhyphen{}SC junction.

\item {} 
\sphinxAtStartPar
\sphinxstyleliteralstrong{\sphinxupquote{L\_b}} (\sphinxstyleliteralemphasis{\sphinxupquote{str}}) \textendash{} Length of the barrier.

\item {} 
\sphinxAtStartPar
\sphinxstyleliteralstrong{\sphinxupquote{v\_b}} (\sphinxstyleliteralemphasis{\sphinxupquote{str}}) \textendash{} Velocity in the barrier.

\item {} 
\sphinxAtStartPar
\sphinxstyleliteralstrong{\sphinxupquote{mu\_b}} (\sphinxstyleliteralemphasis{\sphinxupquote{str}}) \textendash{} Chemical potential in the barrier.

\end{itemize}

\item[{Returns}] \leavevmode
\sphinxAtStartPar
The values of delta\_b and phi\_b.

\item[{Return type}] \leavevmode
\sphinxAtStartPar
array

\end{description}\end{quote}

\end{fulllineitems}

\index{compute\_downstream\_conductance\_TB() (in module modules.utils)@\spxentry{compute\_downstream\_conductance\_TB()}\spxextra{in module modules.utils}}

\begin{fulllineitems}
\phantomsection\label{\detokenize{modules:modules.utils.compute_downstream_conductance_TB}}
\pysigstartsignatures
\pysiglinewithargsret{\sphinxcode{\sphinxupquote{modules.utils.}}\sphinxbfcode{\sphinxupquote{compute\_downstream\_conductance\_TB}}}{\emph{\DUrole{n}{device}}}{}
\pysigstopsignatures
\sphinxAtStartPar
Compute the (zero\sphinxhyphen{}temperature) downstream conductance with Kwant.
\begin{quote}\begin{description}
\item[{Parameters}] \leavevmode
\sphinxAtStartPar
\sphinxstyleliteralstrong{\sphinxupquote{device}} (\sphinxstyleliteralemphasis{\sphinxupquote{object}}) \textendash{} The QH\sphinxhyphen{}SC junction.

\item[{Returns}] \leavevmode
\sphinxAtStartPar
The value of the downstream conductance.

\item[{Return type}] \leavevmode
\sphinxAtStartPar
float

\end{description}\end{quote}

\end{fulllineitems}

\index{compute\_momentum\_difference() (in module modules.utils)@\spxentry{compute\_momentum\_difference()}\spxextra{in module modules.utils}}

\begin{fulllineitems}
\phantomsection\label{\detokenize{modules:modules.utils.compute_momentum_difference}}
\pysigstartsignatures
\pysiglinewithargsret{\sphinxcode{\sphinxupquote{modules.utils.}}\sphinxbfcode{\sphinxupquote{compute\_momentum\_difference}}}{\emph{\DUrole{n}{params}}, \emph{\DUrole{n}{E}}}{}
\pysigstopsignatures
\sphinxAtStartPar
Compute the momentum difference dk.
\begin{quote}\begin{description}
\item[{Parameters}] \leavevmode\begin{itemize}
\item {} 
\sphinxAtStartPar
\sphinxstyleliteralstrong{\sphinxupquote{params}} (\sphinxstyleliteralemphasis{\sphinxupquote{dict}}) \textendash{} The system’s parameters.

\item {} 
\sphinxAtStartPar
\sphinxstyleliteralstrong{\sphinxupquote{E}} (\sphinxstyleliteralemphasis{\sphinxupquote{float}}) \textendash{} The energy value.

\end{itemize}

\item[{Returns}] \leavevmode
\sphinxAtStartPar
The value of dk.

\item[{Return type}] \leavevmode
\sphinxAtStartPar
float

\end{description}\end{quote}

\end{fulllineitems}

\index{compute\_nu\_crit() (in module modules.utils)@\spxentry{compute\_nu\_crit()}\spxextra{in module modules.utils}}

\begin{fulllineitems}
\phantomsection\label{\detokenize{modules:modules.utils.compute_nu_crit}}
\pysigstartsignatures
\pysiglinewithargsret{\sphinxcode{\sphinxupquote{modules.utils.}}\sphinxbfcode{\sphinxupquote{compute\_nu\_crit}}}{\emph{\DUrole{n}{params}}, \emph{\DUrole{n}{nu\_min}\DUrole{o}{=}\DUrole{default_value}{1.0}}, \emph{\DUrole{n}{nu\_max}\DUrole{o}{=}\DUrole{default_value}{3.0}}, \emph{\DUrole{n}{tol}\DUrole{o}{=}\DUrole{default_value}{1e\sphinxhyphen{}06}}}{}
\pysigstopsignatures
\sphinxAtStartPar
Compute the value of nu\_crit.
\begin{quote}\begin{description}
\item[{Parameters}] \leavevmode\begin{itemize}
\item {} 
\sphinxAtStartPar
\sphinxstyleliteralstrong{\sphinxupquote{params}} (\sphinxstyleliteralemphasis{\sphinxupquote{dict}}) \textendash{} The system’s parameters.

\item {} 
\sphinxAtStartPar
\sphinxstyleliteralstrong{\sphinxupquote{nu\_min}} (\sphinxstyleliteralemphasis{\sphinxupquote{float}}) \textendash{} Lower bound, defaults to 1.

\item {} 
\sphinxAtStartPar
\sphinxstyleliteralstrong{\sphinxupquote{nu\_max}} (\sphinxstyleliteralemphasis{\sphinxupquote{float}}) \textendash{} Higher bound, defaults to 3.

\item {} 
\sphinxAtStartPar
\sphinxstyleliteralstrong{\sphinxupquote{tol}} (\sphinxstyleliteralemphasis{\sphinxupquote{float}}) \textendash{} Precision of the returned value, defaults to 1E\sphinxhyphen{}6.

\end{itemize}

\item[{Returns}] \leavevmode
\sphinxAtStartPar
The value of nu\_crit.

\item[{Return type}] \leavevmode
\sphinxAtStartPar
float

\end{description}\end{quote}

\end{fulllineitems}

\index{plot\_density() (in module modules.utils)@\spxentry{plot\_density()}\spxextra{in module modules.utils}}

\begin{fulllineitems}
\phantomsection\label{\detokenize{modules:modules.utils.plot_density}}
\pysigstartsignatures
\pysiglinewithargsret{\sphinxcode{\sphinxupquote{modules.utils.}}\sphinxbfcode{\sphinxupquote{plot\_density}}}{\emph{\DUrole{n}{device}}, \emph{\DUrole{n}{energy}\DUrole{o}{=}\DUrole{default_value}{0.0}}, \emph{\DUrole{n}{fig\_name}\DUrole{o}{=}\DUrole{default_value}{False}}}{}
\pysigstopsignatures
\sphinxAtStartPar
Plot the probability density \(|u|^2 - |v|^2\) of the incoming electron.
\begin{quote}\begin{description}
\item[{Parameters}] \leavevmode\begin{itemize}
\item {} 
\sphinxAtStartPar
\sphinxstyleliteralstrong{\sphinxupquote{device}} (\sphinxstyleliteralemphasis{\sphinxupquote{object}}) \textendash{} The device.

\item {} 
\sphinxAtStartPar
\sphinxstyleliteralstrong{\sphinxupquote{energy}} (\sphinxstyleliteralemphasis{\sphinxupquote{float}}) \textendash{} Value of the energy (relative to the CAES Fermi level), default to 0.

\item {} 
\sphinxAtStartPar
\sphinxstyleliteralstrong{\sphinxupquote{fig\_name}} (\sphinxstyleliteralemphasis{\sphinxupquote{str}}) \textendash{} The name of the plot used for the manuscript, optional.

\end{itemize}

\end{description}\end{quote}

\end{fulllineitems}

\index{plot\_device() (in module modules.utils)@\spxentry{plot\_device()}\spxextra{in module modules.utils}}

\begin{fulllineitems}
\phantomsection\label{\detokenize{modules:modules.utils.plot_device}}
\pysigstartsignatures
\pysiglinewithargsret{\sphinxcode{\sphinxupquote{modules.utils.}}\sphinxbfcode{\sphinxupquote{plot\_device}}}{\emph{\DUrole{n}{device}}}{}
\pysigstopsignatures
\sphinxAtStartPar
Plot the Kwant system.

\sphinxAtStartPar
Be carefull, the colors are badly defined for negative angles.
\begin{quote}\begin{description}
\item[{Parameters}] \leavevmode
\sphinxAtStartPar
\sphinxstyleliteralstrong{\sphinxupquote{device}} (\sphinxstyleliteralemphasis{\sphinxupquote{object}}) \textendash{} The device.

\end{description}\end{quote}

\end{fulllineitems}

\index{plot\_downstream\_conductance\_comparison() (in module modules.utils)@\spxentry{plot\_downstream\_conductance\_comparison()}\spxextra{in module modules.utils}}

\begin{fulllineitems}
\phantomsection\label{\detokenize{modules:modules.utils.plot_downstream_conductance_comparison}}
\pysigstartsignatures
\pysiglinewithargsret{\sphinxcode{\sphinxupquote{modules.utils.}}\sphinxbfcode{\sphinxupquote{plot\_downstream\_conductance\_comparison}}}{\emph{\DUrole{n}{Ls}}, \emph{\DUrole{n}{device}}, \emph{\DUrole{n}{device\_1}}, \emph{\DUrole{n}{device\_2}}, \emph{\DUrole{n}{L\_b}}, \emph{\DUrole{n}{v\_b}}, \emph{\DUrole{n}{mu\_b}}, \emph{\DUrole{n}{fig\_name}\DUrole{o}{=}\DUrole{default_value}{False}}, \emph{\DUrole{n}{from\_data}\DUrole{o}{=}\DUrole{default_value}{True}}}{}
\pysigstopsignatures
\sphinxAtStartPar
Comparison between analytical and tight\sphinxhyphen{}binding conductance.

\sphinxAtStartPar
The analytical formula is shifted in order to recover the tight\sphinxhyphen{}binding simulation at large L.
We we need to give L\_b, v\_b, and mu\_b as inputs.

\sphinxAtStartPar
The shifted data and the plot are saved in the directory ‘files/downstream\_conductance/comparison’.
\begin{quote}\begin{description}
\item[{Parameters}] \leavevmode\begin{itemize}
\item {} 
\sphinxAtStartPar
\sphinxstyleliteralstrong{\sphinxupquote{Ls}} (\sphinxstyleliteralemphasis{\sphinxupquote{list}}) \textendash{} The values of the interface’s length.

\item {} 
\sphinxAtStartPar
\sphinxstyleliteralstrong{\sphinxupquote{device}} (\sphinxstyleliteralemphasis{\sphinxupquote{object}}) \textendash{} The QH\sphinxhyphen{}SC junction.

\item {} 
\sphinxAtStartPar
\sphinxstyleliteralstrong{\sphinxupquote{device\_1}} (\sphinxstyleliteralemphasis{\sphinxupquote{object}}) \textendash{} The first QH\sphinxhyphen{}SC corner.

\item {} 
\sphinxAtStartPar
\sphinxstyleliteralstrong{\sphinxupquote{device\_2}} (\sphinxstyleliteralemphasis{\sphinxupquote{object}}) \textendash{} The second QH\sphinxhyphen{}SC corner.

\item {} 
\sphinxAtStartPar
\sphinxstyleliteralstrong{\sphinxupquote{L\_b}} (\sphinxstyleliteralemphasis{\sphinxupquote{str}}) \textendash{} Length of the barrier.

\item {} 
\sphinxAtStartPar
\sphinxstyleliteralstrong{\sphinxupquote{v\_b}} (\sphinxstyleliteralemphasis{\sphinxupquote{str}}) \textendash{} Velocity in the barrier.

\item {} 
\sphinxAtStartPar
\sphinxstyleliteralstrong{\sphinxupquote{mu\_b}} (\sphinxstyleliteralemphasis{\sphinxupquote{str}}) \textendash{} Chemical potential in the barrier.

\item {} 
\sphinxAtStartPar
\sphinxstyleliteralstrong{\sphinxupquote{fig\_name}} (\sphinxstyleliteralemphasis{\sphinxupquote{str}}) \textendash{} The name of the plot used for the manuscript, optional.

\item {} 
\sphinxAtStartPar
\sphinxstyleliteralstrong{\sphinxupquote{from\_data}} (\sphinxstyleliteralemphasis{\sphinxupquote{bool}}) \textendash{} If True the spectrum is plotted using existing data.
If False the data are computed even if they exist.

\end{itemize}

\end{description}\end{quote}

\end{fulllineitems}

\index{plot\_fh\_p\_vs\_Z\_various\_fillings() (in module modules.utils)@\spxentry{plot\_fh\_p\_vs\_Z\_various\_fillings()}\spxextra{in module modules.utils}}

\begin{fulllineitems}
\phantomsection\label{\detokenize{modules:modules.utils.plot_fh_p_vs_Z_various_fillings}}
\pysigstartsignatures
\pysiglinewithargsret{\sphinxcode{\sphinxupquote{modules.utils.}}\sphinxbfcode{\sphinxupquote{plot\_fh\_p\_vs\_Z\_various\_fillings}}}{\emph{\DUrole{n}{nus}}, \emph{\DUrole{n}{Zs}}, \emph{\DUrole{n}{params}}, \emph{\DUrole{n}{fig\_name}\DUrole{o}{=}\DUrole{default_value}{False}}}{}
\pysigstopsignatures
\sphinxAtStartPar
Plot the hole content f\_h\textasciicircum{}+ vs Z for various values of the filling factor.

\sphinxAtStartPar
The plot is saved in the directory ‘files/andreev\_and\_hole\_prob/hole\_probability/varying\_Z/plots’.
\begin{quote}\begin{description}
\item[{Parameters}] \leavevmode\begin{itemize}
\item {} 
\sphinxAtStartPar
\sphinxstyleliteralstrong{\sphinxupquote{nus}} (\sphinxstyleliteralemphasis{\sphinxupquote{list}}) \textendash{} The values of the filling factor.

\item {} 
\sphinxAtStartPar
\sphinxstyleliteralstrong{\sphinxupquote{Zs}} (\sphinxstyleliteralemphasis{\sphinxupquote{list}}) \textendash{} The values of the barrier strength.

\item {} 
\sphinxAtStartPar
\sphinxstyleliteralstrong{\sphinxupquote{params}} (\sphinxstyleliteralemphasis{\sphinxupquote{dict}}) \textendash{} The system’s parameters.

\item {} 
\sphinxAtStartPar
\sphinxstyleliteralstrong{\sphinxupquote{fig\_name}} (\sphinxstyleliteralemphasis{\sphinxupquote{str}}) \textendash{} The name of the plot used for the manuscript, optional.

\end{itemize}

\end{description}\end{quote}

\end{fulllineitems}

\index{plot\_finite\_T\_conductance\_TB\_vs\_L\_various\_temps() (in module modules.utils)@\spxentry{plot\_finite\_T\_conductance\_TB\_vs\_L\_various\_temps()}\spxextra{in module modules.utils}}

\begin{fulllineitems}
\phantomsection\label{\detokenize{modules:modules.utils.plot_finite_T_conductance_TB_vs_L_various_temps}}
\pysigstartsignatures
\pysiglinewithargsret{\sphinxcode{\sphinxupquote{modules.utils.}}\sphinxbfcode{\sphinxupquote{plot\_finite\_T\_conductance\_TB\_vs\_L\_various\_temps}}}{\emph{\DUrole{n}{device}}, \emph{\DUrole{n}{kTs}}, \emph{\DUrole{n}{Ls}}, \emph{\DUrole{n}{from\_data}\DUrole{o}{=}\DUrole{default_value}{True}}, \emph{\DUrole{n}{fig\_name}\DUrole{o}{=}\DUrole{default_value}{False}}}{}
\pysigstopsignatures
\sphinxAtStartPar
Plot the finite\sphinxhyphen{}temperature downstream conductance versus the energy for various temperatures.

\sphinxAtStartPar
Here we use a full tight\sphinxhyphen{}binding calculation and we compare with the zero\sphinxhyphen{}temperature result.

\sphinxAtStartPar
The data and the plot are saved in the directory
‘files/finite\_temperature/downstream\_conductance/varying\_L/tight\_binding’.
\begin{quote}\begin{description}
\item[{Parameters}] \leavevmode\begin{itemize}
\item {} 
\sphinxAtStartPar
\sphinxstyleliteralstrong{\sphinxupquote{device}} (\sphinxstyleliteralemphasis{\sphinxupquote{object}}) \textendash{} The QH\sphinxhyphen{}SC junction.

\item {} 
\sphinxAtStartPar
\sphinxstyleliteralstrong{\sphinxupquote{kTs}} (\sphinxstyleliteralemphasis{\sphinxupquote{list}}) \textendash{} The values of kB*T.

\item {} 
\sphinxAtStartPar
\sphinxstyleliteralstrong{\sphinxupquote{Ls}} (\sphinxstyleliteralemphasis{\sphinxupquote{list}}) \textendash{} The values of the interface’s length.

\item {} 
\sphinxAtStartPar
\sphinxstyleliteralstrong{\sphinxupquote{from\_data}} (\sphinxstyleliteralemphasis{\sphinxupquote{bool}}) \textendash{} If True the spectrum is plotted using the existing data.
If False the data are computed even if they exist.

\item {} 
\sphinxAtStartPar
\sphinxstyleliteralstrong{\sphinxupquote{fig\_name}} (\sphinxstyleliteralemphasis{\sphinxupquote{str}}) \textendash{} The name of the plot used for the manuscript, optional.

\end{itemize}

\end{description}\end{quote}

\end{fulllineitems}

\index{plot\_k0\_vs\_Z\_various\_fillings() (in module modules.utils)@\spxentry{plot\_k0\_vs\_Z\_various\_fillings()}\spxextra{in module modules.utils}}

\begin{fulllineitems}
\phantomsection\label{\detokenize{modules:modules.utils.plot_k0_vs_Z_various_fillings}}
\pysigstartsignatures
\pysiglinewithargsret{\sphinxcode{\sphinxupquote{modules.utils.}}\sphinxbfcode{\sphinxupquote{plot\_k0\_vs\_Z\_various\_fillings}}}{\emph{\DUrole{n}{nus}}, \emph{\DUrole{n}{Zs}}, \emph{\DUrole{n}{params}}, \emph{\DUrole{n}{fig\_name}\DUrole{o}{=}\DUrole{default_value}{False}}}{}
\pysigstopsignatures
\sphinxAtStartPar
Plot the momentum k0 versus Z for various values of the filling factor.

\sphinxAtStartPar
The plot is saved in the directory ‘files/momentum\_k0/varying\_Z/plots’.
\begin{quote}\begin{description}
\item[{Parameters}] \leavevmode\begin{itemize}
\item {} 
\sphinxAtStartPar
\sphinxstyleliteralstrong{\sphinxupquote{nus}} (\sphinxstyleliteralemphasis{\sphinxupquote{list}}) \textendash{} The values of the filling factor.

\item {} 
\sphinxAtStartPar
\sphinxstyleliteralstrong{\sphinxupquote{Zs}} (\sphinxstyleliteralemphasis{\sphinxupquote{list}}) \textendash{} The values of the barrier strength.

\item {} 
\sphinxAtStartPar
\sphinxstyleliteralstrong{\sphinxupquote{params}} (\sphinxstyleliteralemphasis{\sphinxupquote{dict}}) \textendash{} The system’s parameters.

\item {} 
\sphinxAtStartPar
\sphinxstyleliteralstrong{\sphinxupquote{fig\_name}} (\sphinxstyleliteralemphasis{\sphinxupquote{str}}) \textendash{} The name of the plot used for the manuscript, optional.

\end{itemize}

\end{description}\end{quote}

\end{fulllineitems}

\index{plot\_k0\_vs\_nu() (in module modules.utils)@\spxentry{plot\_k0\_vs\_nu()}\spxextra{in module modules.utils}}

\begin{fulllineitems}
\phantomsection\label{\detokenize{modules:modules.utils.plot_k0_vs_nu}}
\pysigstartsignatures
\pysiglinewithargsret{\sphinxcode{\sphinxupquote{modules.utils.}}\sphinxbfcode{\sphinxupquote{plot\_k0\_vs\_nu}}}{\emph{\DUrole{n}{nus}}, \emph{\DUrole{n}{params}}, \emph{\DUrole{n}{fig\_name}\DUrole{o}{=}\DUrole{default_value}{False}}}{}
\pysigstopsignatures
\sphinxAtStartPar
Plot the momentum at the Fermi level k0 versus nu.

\sphinxAtStartPar
The plot is saved in the directory ‘files/momentum\_k0/varying\_nu/plots’.
\begin{quote}\begin{description}
\item[{Parameters}] \leavevmode\begin{itemize}
\item {} 
\sphinxAtStartPar
\sphinxstyleliteralstrong{\sphinxupquote{nus}} (\sphinxstyleliteralemphasis{\sphinxupquote{list}}) \textendash{} The values of the filling factor.

\item {} 
\sphinxAtStartPar
\sphinxstyleliteralstrong{\sphinxupquote{params}} (\sphinxstyleliteralemphasis{\sphinxupquote{dict}}) \textendash{} The system’s parameters.

\item {} 
\sphinxAtStartPar
\sphinxstyleliteralstrong{\sphinxupquote{fig\_name}} (\sphinxstyleliteralemphasis{\sphinxupquote{str}}) \textendash{} The name of the plot used for the manuscript, optional.

\end{itemize}

\end{description}\end{quote}

\end{fulllineitems}

\index{plot\_momentum\_difference\_vs\_energy() (in module modules.utils)@\spxentry{plot\_momentum\_difference\_vs\_energy()}\spxextra{in module modules.utils}}

\begin{fulllineitems}
\phantomsection\label{\detokenize{modules:modules.utils.plot_momentum_difference_vs_energy}}
\pysigstartsignatures
\pysiglinewithargsret{\sphinxcode{\sphinxupquote{modules.utils.}}\sphinxbfcode{\sphinxupquote{plot\_momentum\_difference\_vs\_energy}}}{\emph{\DUrole{n}{params}}, \emph{\DUrole{n}{fig\_name}\DUrole{o}{=}\DUrole{default_value}{False}}}{}
\pysigstopsignatures
\sphinxAtStartPar
Plot momentum difference \sphinxstyleemphasis{v.s.} energy.

\sphinxAtStartPar
The resulting plot is saved in the ‘files/finite\_temperature/momentum\_difference/varying\_energy’ directory.
\begin{quote}\begin{description}
\item[{Parameters}] \leavevmode\begin{itemize}
\item {} 
\sphinxAtStartPar
\sphinxstyleliteralstrong{\sphinxupquote{params}} (\sphinxstyleliteralemphasis{\sphinxupquote{dict}}) \textendash{} The system’s parameters.

\item {} 
\sphinxAtStartPar
\sphinxstyleliteralstrong{\sphinxupquote{fig\_name}} (\sphinxstyleliteralemphasis{\sphinxupquote{str}}) \textendash{} The name of the plot used for the manuscript, optional.

\end{itemize}

\end{description}\end{quote}

\end{fulllineitems}

\index{plot\_nu\_crit\_limit\_vs\_Z() (in module modules.utils)@\spxentry{plot\_nu\_crit\_limit\_vs\_Z()}\spxextra{in module modules.utils}}

\begin{fulllineitems}
\phantomsection\label{\detokenize{modules:modules.utils.plot_nu_crit_limit_vs_Z}}
\pysigstartsignatures
\pysiglinewithargsret{\sphinxcode{\sphinxupquote{modules.utils.}}\sphinxbfcode{\sphinxupquote{plot\_nu\_crit\_limit\_vs\_Z}}}{\emph{\DUrole{n}{params}}, \emph{\DUrole{n}{from\_data}\DUrole{o}{=}\DUrole{default_value}{True}}, \emph{\DUrole{n}{fig\_name}\DUrole{o}{=}\DUrole{default_value}{False}}}{}
\pysigstopsignatures
\sphinxAtStartPar
Plot the asymptotic value of nu\_crit \sphinxstyleemphasis{v.s.} Z.

\sphinxAtStartPar
The resulting plot is saved in the ‘files/track\_states/varying\_Z’ directory.
\begin{quote}\begin{description}
\item[{Parameters}] \leavevmode\begin{itemize}
\item {} 
\sphinxAtStartPar
\sphinxstyleliteralstrong{\sphinxupquote{params}} (\sphinxstyleliteralemphasis{\sphinxupquote{dict}}) \textendash{} The system’s parameters.

\item {} 
\sphinxAtStartPar
\sphinxstyleliteralstrong{\sphinxupquote{from\_data}} (\sphinxstyleliteralemphasis{\sphinxupquote{bool}}) \textendash{} If True the spectrum is plotted using the existing data.
If False the data are computed even if they exist. Defaults to True.

\item {} 
\sphinxAtStartPar
\sphinxstyleliteralstrong{\sphinxupquote{fig\_name}} (\sphinxstyleliteralemphasis{\sphinxupquote{str}}) \textendash{} The name of the plot used for the manuscript, optional.

\end{itemize}

\end{description}\end{quote}

\end{fulllineitems}

\index{plot\_nu\_crit\_limit\_vs\_mismatch() (in module modules.utils)@\spxentry{plot\_nu\_crit\_limit\_vs\_mismatch()}\spxextra{in module modules.utils}}

\begin{fulllineitems}
\phantomsection\label{\detokenize{modules:modules.utils.plot_nu_crit_limit_vs_mismatch}}
\pysigstartsignatures
\pysiglinewithargsret{\sphinxcode{\sphinxupquote{modules.utils.}}\sphinxbfcode{\sphinxupquote{plot\_nu\_crit\_limit\_vs\_mismatch}}}{\emph{\DUrole{n}{params}}, \emph{\DUrole{n}{from\_data}\DUrole{o}{=}\DUrole{default_value}{True}}, \emph{\DUrole{n}{fig\_name}\DUrole{o}{=}\DUrole{default_value}{False}}}{}
\pysigstopsignatures
\sphinxAtStartPar
Plot the asymptotic value of nu\_crit \sphinxstyleemphasis{v.s.} mu\_sc/mu\_qh.

\sphinxAtStartPar
The resulting plot is saved in the ‘files/track\_states/varying\_mismatch’ directory.
\begin{quote}\begin{description}
\item[{Parameters}] \leavevmode\begin{itemize}
\item {} 
\sphinxAtStartPar
\sphinxstyleliteralstrong{\sphinxupquote{params}} (\sphinxstyleliteralemphasis{\sphinxupquote{dict}}) \textendash{} The system’s parameters.

\item {} 
\sphinxAtStartPar
\sphinxstyleliteralstrong{\sphinxupquote{from\_data}} (\sphinxstyleliteralemphasis{\sphinxupquote{bool}}) \textendash{} If True the spectrum is plotted using the existing data.
If False the data are computed even if they exist. Defaults to True.

\item {} 
\sphinxAtStartPar
\sphinxstyleliteralstrong{\sphinxupquote{fig\_name}} (\sphinxstyleliteralemphasis{\sphinxupquote{str}}) \textendash{} The name of the plot used for the manuscript, optional.

\end{itemize}

\end{description}\end{quote}

\end{fulllineitems}

\index{plot\_nu\_crit\_vs\_mu\_qh\_delta() (in module modules.utils)@\spxentry{plot\_nu\_crit\_vs\_mu\_qh\_delta()}\spxextra{in module modules.utils}}

\begin{fulllineitems}
\phantomsection\label{\detokenize{modules:modules.utils.plot_nu_crit_vs_mu_qh_delta}}
\pysigstartsignatures
\pysiglinewithargsret{\sphinxcode{\sphinxupquote{modules.utils.}}\sphinxbfcode{\sphinxupquote{plot\_nu\_crit\_vs\_mu\_qh\_delta}}}{\emph{\DUrole{n}{params}}, \emph{\DUrole{n}{from\_data}\DUrole{o}{=}\DUrole{default_value}{True}}, \emph{\DUrole{n}{fig\_name}\DUrole{o}{=}\DUrole{default_value}{False}}}{}
\pysigstopsignatures
\sphinxAtStartPar
Plot nu\_crit \sphinxstyleemphasis{v.s.} mu\_qh/delta.

\sphinxAtStartPar
The resulting plot is saved in the ‘files/track\_states/varying\_mu\_qh\_delta’ directory.
\begin{quote}\begin{description}
\item[{Parameters}] \leavevmode\begin{itemize}
\item {} 
\sphinxAtStartPar
\sphinxstyleliteralstrong{\sphinxupquote{params}} (\sphinxstyleliteralemphasis{\sphinxupquote{dict}}) \textendash{} The system’s parameters.

\item {} 
\sphinxAtStartPar
\sphinxstyleliteralstrong{\sphinxupquote{from\_data}} (\sphinxstyleliteralemphasis{\sphinxupquote{bool}}) \textendash{} If True the spectrum is plotted using the existing data.
If False the data are computed even if they exist. Defaults to True.

\item {} 
\sphinxAtStartPar
\sphinxstyleliteralstrong{\sphinxupquote{fig\_name}} (\sphinxstyleliteralemphasis{\sphinxupquote{str}}) \textendash{} The name of the plot used for the manuscript, optional.

\end{itemize}

\end{description}\end{quote}

\end{fulllineitems}

\index{plot\_spectrum\_TB() (in module modules.utils)@\spxentry{plot\_spectrum\_TB()}\spxextra{in module modules.utils}}

\begin{fulllineitems}
\phantomsection\label{\detokenize{modules:modules.utils.plot_spectrum_TB}}
\pysigstartsignatures
\pysiglinewithargsret{\sphinxcode{\sphinxupquote{modules.utils.}}\sphinxbfcode{\sphinxupquote{plot\_spectrum\_TB}}}{\emph{\DUrole{n}{device}}, \emph{\DUrole{n}{from\_data}\DUrole{o}{=}\DUrole{default_value}{True}}}{}
\pysigstopsignatures
\sphinxAtStartPar
Plot tight\sphinxhyphen{}binding spectrum.

\sphinxAtStartPar
The data and plots are saved in the directory ‘files/energy\_spectrum/tight\_binding’.
\begin{quote}\begin{description}
\item[{Parameters}] \leavevmode\begin{itemize}
\item {} 
\sphinxAtStartPar
\sphinxstyleliteralstrong{\sphinxupquote{device}} (\sphinxstyleliteralemphasis{\sphinxupquote{object}}) \textendash{} Infinite QH\sphinxhyphen{}SC interface.

\item {} 
\sphinxAtStartPar
\sphinxstyleliteralstrong{\sphinxupquote{from\_data}} (\sphinxstyleliteralemphasis{\sphinxupquote{bool}}) \textendash{} If true the spectrum is plotted using the existing data.
If False the data are computed even if they exist.

\end{itemize}

\end{description}\end{quote}

\end{fulllineitems}

\index{plot\_spectrum\_comparison() (in module modules.utils)@\spxentry{plot\_spectrum\_comparison()}\spxextra{in module modules.utils}}

\begin{fulllineitems}
\phantomsection\label{\detokenize{modules:modules.utils.plot_spectrum_comparison}}
\pysigstartsignatures
\pysiglinewithargsret{\sphinxcode{\sphinxupquote{modules.utils.}}\sphinxbfcode{\sphinxupquote{plot\_spectrum\_comparison}}}{\emph{\DUrole{n}{device}}, \emph{\DUrole{n}{params}}, \emph{\DUrole{n}{fig\_name}\DUrole{o}{=}\DUrole{default_value}{False}}, \emph{\DUrole{n}{from\_data}\DUrole{o}{=}\DUrole{default_value}{True}}}{}
\pysigstopsignatures
\sphinxAtStartPar
Comparison between microscopic and tight\sphinxhyphen{}binding spectrums.

\sphinxAtStartPar
The plots are saved in the directory ‘files/energy\_spectrum/comparison/plots’.
\begin{quote}\begin{description}
\item[{Parameters}] \leavevmode\begin{itemize}
\item {} 
\sphinxAtStartPar
\sphinxstyleliteralstrong{\sphinxupquote{device}} (\sphinxstyleliteralemphasis{\sphinxupquote{object}}) \textendash{} Infinite QH\sphinxhyphen{}SC interface.

\item {} 
\sphinxAtStartPar
\sphinxstyleliteralstrong{\sphinxupquote{params}} (\sphinxstyleliteralemphasis{\sphinxupquote{dict}}) \textendash{} The system’s parameters.

\item {} 
\sphinxAtStartPar
\sphinxstyleliteralstrong{\sphinxupquote{fig\_name}} (\sphinxstyleliteralemphasis{\sphinxupquote{str}}) \textendash{} The name of the plot used for the manuscript, optional.

\item {} 
\sphinxAtStartPar
\sphinxstyleliteralstrong{\sphinxupquote{from\_data}} (\sphinxstyleliteralemphasis{\sphinxupquote{bool}}) \textendash{} If true the spectrum is plotted using the existing data.
If False the data are computed even if they exist.

\end{itemize}

\end{description}\end{quote}

\end{fulllineitems}

\index{plot\_spectrum\_micro() (in module modules.utils)@\spxentry{plot\_spectrum\_micro()}\spxextra{in module modules.utils}}

\begin{fulllineitems}
\phantomsection\label{\detokenize{modules:modules.utils.plot_spectrum_micro}}
\pysigstartsignatures
\pysiglinewithargsret{\sphinxcode{\sphinxupquote{modules.utils.}}\sphinxbfcode{\sphinxupquote{plot\_spectrum\_micro}}}{\emph{\DUrole{n}{params}}, \emph{\DUrole{n}{from\_data}\DUrole{o}{=}\DUrole{default_value}{True}}, \emph{\DUrole{n}{fig\_name}\DUrole{o}{=}\DUrole{default_value}{False}}, \emph{\DUrole{n}{show\_k0}\DUrole{o}{=}\DUrole{default_value}{False}}, \emph{\DUrole{n}{qp\_labels}\DUrole{o}{=}\DUrole{default_value}{False}}}{}
\pysigstopsignatures
\sphinxAtStartPar
Plot microscopic spectrum.

\sphinxAtStartPar
The data and plots are saved in the directory ‘files/energy\_spectrum/microscopic’.
\begin{quote}\begin{description}
\item[{Parameters}] \leavevmode\begin{itemize}
\item {} 
\sphinxAtStartPar
\sphinxstyleliteralstrong{\sphinxupquote{params}} (\sphinxstyleliteralemphasis{\sphinxupquote{dict}}) \textendash{} The system’s parameters.

\item {} 
\sphinxAtStartPar
\sphinxstyleliteralstrong{\sphinxupquote{from\_data}} (\sphinxstyleliteralemphasis{\sphinxupquote{bool}}) \textendash{} If true the spectrum is plotted using the existing data.
If False the data are computed even if they exist.

\item {} 
\sphinxAtStartPar
\sphinxstyleliteralstrong{\sphinxupquote{fig\_name}} (\sphinxstyleliteralemphasis{\sphinxupquote{str}}) \textendash{} The name of the plot used for the manuscript, optional.

\item {} 
\sphinxAtStartPar
\sphinxstyleliteralstrong{\sphinxupquote{show\_k0}} (\sphinxstyleliteralemphasis{\sphinxupquote{str}}) \textendash{} Show the position of k0, default to False.

\item {} 
\sphinxAtStartPar
\sphinxstyleliteralstrong{\sphinxupquote{qp\_labels}} (\sphinxstyleliteralemphasis{\sphinxupquote{str}}) \textendash{} Show the quasiparticles labels, default to False.

\end{itemize}

\end{description}\end{quote}

\end{fulllineitems}

\index{plot\_tau\_vs\_energy() (in module modules.utils)@\spxentry{plot\_tau\_vs\_energy()}\spxextra{in module modules.utils}}

\begin{fulllineitems}
\phantomsection\label{\detokenize{modules:modules.utils.plot_tau_vs_energy}}
\pysigstartsignatures
\pysiglinewithargsret{\sphinxcode{\sphinxupquote{modules.utils.}}\sphinxbfcode{\sphinxupquote{plot\_tau\_vs\_energy}}}{\emph{\DUrole{n}{device}}, \emph{\DUrole{n}{fig\_name}\DUrole{o}{=}\DUrole{default_value}{False}}, \emph{\DUrole{n}{from\_data}\DUrole{o}{=}\DUrole{default_value}{True}}, \emph{\DUrole{n}{tau\_label}\DUrole{o}{=}\DUrole{default_value}{None}}}{}
\pysigstopsignatures
\sphinxAtStartPar
Plot the conversion probabilty tau vs energy.

\sphinxAtStartPar
The data and the plot are saved in the directory
‘files/finite\_temperature/scattering\_probabilities/transmissions/’.
\begin{quote}\begin{description}
\item[{Parameters}] \leavevmode\begin{itemize}
\item {} 
\sphinxAtStartPar
\sphinxstyleliteralstrong{\sphinxupquote{device}} (\sphinxstyleliteralemphasis{\sphinxupquote{object}}) \textendash{} The QH\sphinxhyphen{}SC junction.

\item {} 
\sphinxAtStartPar
\sphinxstyleliteralstrong{\sphinxupquote{fig\_name}} (\sphinxstyleliteralemphasis{\sphinxupquote{str}}) \textendash{} The name of the plot used for the manuscript, optional.

\item {} 
\sphinxAtStartPar
\sphinxstyleliteralstrong{\sphinxupquote{from\_data}} (\sphinxstyleliteralemphasis{\sphinxupquote{bool}}) \textendash{} If True the spectrum is plotted using the existing data.
If False the data are computed even if they exist. Defaults to True.

\item {} 
\sphinxAtStartPar
\sphinxstyleliteralstrong{\sphinxupquote{tau\_label}} (\sphinxstyleliteralemphasis{\sphinxupquote{int}}) \textendash{} None, 1 or 2. Defaults to None.

\end{itemize}

\end{description}\end{quote}

\end{fulllineitems}

\index{plot\_tau\_vs\_mu\_qh\_delta\_various\_fillings() (in module modules.utils)@\spxentry{plot\_tau\_vs\_mu\_qh\_delta\_various\_fillings()}\spxextra{in module modules.utils}}

\begin{fulllineitems}
\phantomsection\label{\detokenize{modules:modules.utils.plot_tau_vs_mu_qh_delta_various_fillings}}
\pysigstartsignatures
\pysiglinewithargsret{\sphinxcode{\sphinxupquote{modules.utils.}}\sphinxbfcode{\sphinxupquote{plot\_tau\_vs\_mu\_qh\_delta\_various\_fillings}}}{\emph{\DUrole{n}{nus}}, \emph{\DUrole{n}{deltas}}, \emph{\DUrole{n}{theta\_qh}}, \emph{\DUrole{n}{theta\_sc}}, \emph{\DUrole{n}{params}}, \emph{\DUrole{n}{from\_data}\DUrole{o}{=}\DUrole{default_value}{True}}, \emph{\DUrole{n}{fig\_name}\DUrole{o}{=}\DUrole{default_value}{False}}}{}
\pysigstopsignatures
\sphinxAtStartPar
Plot the corner’s Andreev transmission versus mu\_\{QH\}/Delta for various values of the filling factor.

\sphinxAtStartPar
The data and plot are saved in the directory 
‘files/chapter3/andreev\_transmission/varying\_mu\_qh\_delta’.
\begin{quote}\begin{description}
\item[{Parameters}] \leavevmode\begin{itemize}
\item {} 
\sphinxAtStartPar
\sphinxstyleliteralstrong{\sphinxupquote{nus}} (\sphinxstyleliteralemphasis{\sphinxupquote{list}}) \textendash{} The values of the filling factor.

\item {} 
\sphinxAtStartPar
\sphinxstyleliteralstrong{\sphinxupquote{deltas}} (\sphinxstyleliteralemphasis{\sphinxupquote{list}}) \textendash{} The values of the SC gap.

\item {} 
\sphinxAtStartPar
\sphinxstyleliteralstrong{\sphinxupquote{theta\_qh}} (\sphinxstyleliteralemphasis{\sphinxupquote{float}}) \textendash{} The QH angle.

\item {} 
\sphinxAtStartPar
\sphinxstyleliteralstrong{\sphinxupquote{theta\_sc}} (\sphinxstyleliteralemphasis{\sphinxupquote{float}}) \textendash{} The SC angle.

\item {} 
\sphinxAtStartPar
\sphinxstyleliteralstrong{\sphinxupquote{params}} (\sphinxstyleliteralemphasis{\sphinxupquote{dict}}) \textendash{} The system’s parameters.

\item {} 
\sphinxAtStartPar
\sphinxstyleliteralstrong{\sphinxupquote{from\_data}} (\sphinxstyleliteralemphasis{\sphinxupquote{bool}}) \textendash{} If True the spectrum is plotted using the existing data.
If False the data are computed even if they exist.

\item {} 
\sphinxAtStartPar
\sphinxstyleliteralstrong{\sphinxupquote{fig\_name}} (\sphinxstyleliteralemphasis{\sphinxupquote{str}}) \textendash{} The name of the plot used for the manuscript, optional.

\end{itemize}

\end{description}\end{quote}

\end{fulllineitems}

\index{plot\_tau\_vs\_theta\_qh\_various\_fillings() (in module modules.utils)@\spxentry{plot\_tau\_vs\_theta\_qh\_various\_fillings()}\spxextra{in module modules.utils}}

\begin{fulllineitems}
\phantomsection\label{\detokenize{modules:modules.utils.plot_tau_vs_theta_qh_various_fillings}}
\pysigstartsignatures
\pysiglinewithargsret{\sphinxcode{\sphinxupquote{modules.utils.}}\sphinxbfcode{\sphinxupquote{plot\_tau\_vs\_theta\_qh\_various\_fillings}}}{\emph{\DUrole{n}{nus}}, \emph{\DUrole{n}{thetas}}, \emph{\DUrole{n}{device}}, \emph{\DUrole{n}{fig\_name}\DUrole{o}{=}\DUrole{default_value}{False}}, \emph{\DUrole{n}{from\_data}\DUrole{o}{=}\DUrole{default_value}{True}}, \emph{\DUrole{n}{show\_only\_commensurate}\DUrole{o}{=}\DUrole{default_value}{False}}}{}
\pysigstopsignatures
\sphinxAtStartPar
Plot the corner’s Andreev transmission versus the QH angle for various values of the filling factor.

\sphinxAtStartPar
The data and the plot are saved in the directory 
‘files/andreev\_and\_hole\_prob/andreev\_transmission/varying\_theta’.
\begin{quote}\begin{description}
\item[{Parameters}] \leavevmode\begin{itemize}
\item {} 
\sphinxAtStartPar
\sphinxstyleliteralstrong{\sphinxupquote{nus}} (\sphinxstyleliteralemphasis{\sphinxupquote{list}}) \textendash{} The values of the filling factor.

\item {} 
\sphinxAtStartPar
\sphinxstyleliteralstrong{\sphinxupquote{thetas}} (\sphinxstyleliteralemphasis{\sphinxupquote{list}}) \textendash{} The values of the QH angle.

\item {} 
\sphinxAtStartPar
\sphinxstyleliteralstrong{\sphinxupquote{params}} (\sphinxstyleliteralemphasis{\sphinxupquote{dict}}) \textendash{} The system’s parameters.

\item {} 
\sphinxAtStartPar
\sphinxstyleliteralstrong{\sphinxupquote{fig\_name}} (\sphinxstyleliteralemphasis{\sphinxupquote{str}}) \textendash{} The name of the plot used for the manuscript, optional.

\item {} 
\sphinxAtStartPar
\sphinxstyleliteralstrong{\sphinxupquote{from\_data}} (\sphinxstyleliteralemphasis{\sphinxupquote{bool}}) \textendash{} If True the spectrum is plotted using the stored data.
If False the data are computed even if they exist.

\item {} 
\sphinxAtStartPar
\sphinxstyleliteralstrong{\sphinxupquote{show\_only\_commensurate}} (\sphinxstyleliteralemphasis{\sphinxupquote{bool}}) \textendash{} If True only the commensurate angles are shown. Defaults to False.

\end{itemize}

\end{description}\end{quote}

\end{fulllineitems}

\index{plot\_tau\_vs\_theta\_sc\_various\_fillings() (in module modules.utils)@\spxentry{plot\_tau\_vs\_theta\_sc\_various\_fillings()}\spxextra{in module modules.utils}}

\begin{fulllineitems}
\phantomsection\label{\detokenize{modules:modules.utils.plot_tau_vs_theta_sc_various_fillings}}
\pysigstartsignatures
\pysiglinewithargsret{\sphinxcode{\sphinxupquote{modules.utils.}}\sphinxbfcode{\sphinxupquote{plot\_tau\_vs\_theta\_sc\_various\_fillings}}}{\emph{\DUrole{n}{nus}}, \emph{\DUrole{n}{thetas}}, \emph{\DUrole{n}{device}}, \emph{\DUrole{n}{fig\_name}\DUrole{o}{=}\DUrole{default_value}{False}}, \emph{\DUrole{n}{from\_data}\DUrole{o}{=}\DUrole{default_value}{True}}, \emph{\DUrole{n}{show\_only\_commensurate}\DUrole{o}{=}\DUrole{default_value}{False}}}{}
\pysigstopsignatures
\sphinxAtStartPar
Plot the corner’s Andreev transmission versus the SC angle for various values of the filling factor.

\sphinxAtStartPar
The data and the plot are saved in the directory 
‘files/andreev\_and\_hole\_prob/andreev\_transmission/varying\_theta\_sc’.
\begin{quote}\begin{description}
\item[{Parameters}] \leavevmode\begin{itemize}
\item {} 
\sphinxAtStartPar
\sphinxstyleliteralstrong{\sphinxupquote{nus}} (\sphinxstyleliteralemphasis{\sphinxupquote{list}}) \textendash{} The values of the filling factor.

\item {} 
\sphinxAtStartPar
\sphinxstyleliteralstrong{\sphinxupquote{thetas}} (\sphinxstyleliteralemphasis{\sphinxupquote{list}}) \textendash{} The values of the SC angle.

\item {} 
\sphinxAtStartPar
\sphinxstyleliteralstrong{\sphinxupquote{params}} (\sphinxstyleliteralemphasis{\sphinxupquote{dict}}) \textendash{} The system’s parameters.

\item {} 
\sphinxAtStartPar
\sphinxstyleliteralstrong{\sphinxupquote{fig\_name}} (\sphinxstyleliteralemphasis{\sphinxupquote{str}}) \textendash{} The name of the plot used for the manuscript, optional.

\item {} 
\sphinxAtStartPar
\sphinxstyleliteralstrong{\sphinxupquote{from\_data}} (\sphinxstyleliteralemphasis{\sphinxupquote{bool}}) \textendash{} If True the spectrum is plotted using the stored data.
If False the data are computed even if they exist.

\item {} 
\sphinxAtStartPar
\sphinxstyleliteralstrong{\sphinxupquote{show\_only\_commensurate}} (\sphinxstyleliteralemphasis{\sphinxupquote{bool}}) \textendash{} If True only the commensurate angles are shown. Defaults to False.

\end{itemize}

\end{description}\end{quote}

\end{fulllineitems}



\chapter{Indices and tables}
\label{\detokenize{index:indices-and-tables}}\begin{itemize}
\item {} 
\sphinxAtStartPar
\DUrole{xref,std,std-ref}{genindex}

\item {} 
\sphinxAtStartPar
\DUrole{xref,std,std-ref}{modindex}

\item {} 
\sphinxAtStartPar
\DUrole{xref,std,std-ref}{search}

\end{itemize}


\chapter{License}
\label{\detokenize{index:license}}
\sphinxAtStartPar
See the \sphinxhref{https://github.com/akdavid/2deg\_QH-SC/blob/main/LICENCE.rst}{license terms}
for license rights and limitations (MIT).


\renewcommand{\indexname}{Python Module Index}
\begin{sphinxtheindex}
\let\bigletter\sphinxstyleindexlettergroup
\bigletter{c}
\item\relax\sphinxstyleindexentry{calculations}\sphinxstyleindexpageref{main_scripts:\detokenize{module-calculations}}
\indexspace
\bigletter{m}
\item\relax\sphinxstyleindexentry{manuscript\_figures}\sphinxstyleindexpageref{main_scripts:\detokenize{module-manuscript_figures}}
\item\relax\sphinxstyleindexentry{modules.functions}\sphinxstyleindexpageref{modules:\detokenize{module-modules.functions}}
\item\relax\sphinxstyleindexentry{modules.system}\sphinxstyleindexpageref{modules:\detokenize{module-modules.system}}
\item\relax\sphinxstyleindexentry{modules.utils}\sphinxstyleindexpageref{modules:\detokenize{module-modules.utils}}
\end{sphinxtheindex}

\renewcommand{\indexname}{Index}
\printindex
\end{document}